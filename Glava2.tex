
\newpage 

%%_________________________________________________________________________________________________________________________
%%_________________________________________________________________________________________________________________________
%%_________________________________________________________________________________________________________________________

\subsection{Турбинная камера}


Турбинная камера служит для подвода воды к направляющему аппарату реактивной турбины. К турбинным камерам предъявляются следующие требования:

\begin{description}
\item[1.] Они должны обеспечивать равномерное по всему периметру питание направляющего аппарата.
\item[2.] Гидравлические потери в самой камере при обтекании статорных колонн и при входе потока в направляющий аппарат должны быть минимальными.
\item[3.] Форма и размеры турбинной камеры должны соответствовать условиям компоновки здания ГЭС и позволять удобное сопряжение с напорными водоводами ГЭС.
\end{description}


Для подвода воды к горизонтальным турбинам применяются прямоосные турбинные камеры. Для подвода воды к турбинам с вертикальной осью вращения применяют спиральные камеры.

Основными характеристиками спиральной камеры являются угол охвата $\varphi_{\text{охв}}$ и ширина спиральной камеры $B$. Также спиральную камеру характеризует форма её поперечного сечения.

\begin{figure} [ht]
  \center
  \includegraphics [scale = 0.9] {ppm}
  \caption{Габаритные характеристики спиральных камер. \\ \textit{а. Бетонная спиральная камера таврового поперечного сечения;\\ б. Металлическая спиральная камера круглого сечения.} }
  \label{img_ppm}
\end{figure}

Поперечными сечениями спиральной камеры называют сечения, образованные плоскостями, проходящими через ось вращения турбины.

Входным сечением спиральной камеры считается сечение, перпендикулярное оси подводящего водовода. Концевое сечение принято определять по выходной кромке замыкающей колонны статора -- <<зуба спирали>>.

\begin{figure} [ht]
  \center
  \includegraphics [scale = 0.6] {interc}
  \caption{Монтаж металлической спиральной камеры.}
  \label{img_interc}
\end{figure}

%%_________________________________________________________________________________________________________________________
%%_________________________________________________________________________________________________________________________
%%_________________________________________________________________________________________________________________________

\subsubsection{Типы спиральных камер.}

Для подвода воды к крупным турбинам применяются, в основном, два типа спиральных камер: бетонные (железобетонные), имеющие как правило трапецеидальное (тавровое) поперечное сечение, и металлические (стальные) круглого поперечного сечения. Для мелких турбин иногда используются упрощённые типы спиральных камер: открытые, прямоугольные и кожуховые.

Области использования различных типов турбинных камер определяются, главным образом, величиной напора $H$.

\begin{description}
\item[-] бетонные: $H$ от 4 до 50 м; 
\item[-] бетонные с металлической облицовкой: $H$ от 40 до 80 м;
\item[-] металлические: $H$ от 40 до 700 м;
\item[-] сталежелезобетонные: $H$ от 100 до 300 м и больших величинах расхода $Q_{\text{г.а.}}$
\end{description}







\textbf{Бетонные спиральные камеры} имеют трапецеидальное поперечное сечение, площадь которого постепенно уменьшается от входа в спиральную камеру к её концу. Такое уменьшение сечения  необходимо для равномерного распределения расхода воды, подводимого на турбину. Величина угла охвата $\varphi_{\text{охв}}$ зависит от напора и принимается в пределах $\varphi_{\text{охв}} = 180 - 270^\circ$.

\begin{figure} [ht]
  \center
  \includegraphics [scale = 0.9] {LogoAsf}
  \caption{Зависимость величины $\varphi_{\text{охв}}$ от напора $H$.}
  \label{img_ppn}
\end{figure}


Одна и существенных особенностей бетонных спиральных камер состоит в том, что значительная часть периметра направляющего аппарата (см. рис. \ref{img_ppm}, а -- часть периметра между  точками 1 и 2) получает воду непосредственно из подводящего водовода (при $\varphi_{\text{охв}} = 180^\circ$ -- половина). Для улучшения условий входа потока в направляющий аппарат в этой части спиральной камеры устанавливают криволинейные колонны (статорные колонны), причём их шаг здесь делают меньше.

Форма поперечных сечений спиральной камеры определяется условиями наивыгоднейшей компоновки здания ГЭС. Различные формы поперечных сечений спиральных камер показаны на рисунке \ref{img_ppo}. Они отличаются по соотношению размеров $b_1$ и $b_2$. Помимо этих размеров форму поперечных сечений также характеризуют следующие величины:

$b_0$ -- высота направляющего аппарата;
  
$b$ -- высота поперечного сечения спиральной камеры $b = b_1 + b_2 + b_0$;

$R$ -- величина, характеризующая ширину поперечного сечения спиральной камеры;

$R_b$ -- радиус окружности, проведённой по \textbf{выходным} кромкам статорных колонн, $$R_b = \frac{D_b}{2} \; ;$$ 

$R_a$ -- радиус окружности, проведённой по \textbf{входным} кромкам статорных колонн, $$R_a = \frac{D_a}{2} \; ;$$

$\alpha_1$ -- угол наклона верхней части спиральной камеры при переходе к статору, $$\alpha_1 = 25 - 30^\circ \; ;$$

$\alpha_2$ -- угол наклона нижней части спиральной камеры при переходе к статору, \\ $$\alpha_2 = 15 - 25^\circ \; .$$


\begin{figure} [ht]
  \center
  \includegraphics [scale = 0.9] {LogoAsf}
  \caption{Формы тавровых сечений бетонных спиральных камер. \\ \textit{а. Симметричного сечения $b_1 = b_2$; б. Развитые вверх $b_1 > b_2$; в. Развитые вниз $b_1 < b_2$; г. С горизонтальным потолком $b_1 = 0$; д. С горизонтальным полом $b_2 = 0$.}}
  \label{img_ppo}
\end{figure}



\vspace{0.5cm}

Поскольку турбинная камера непосредственно примыкает к статорному кольцу, для её проектирования необходимо знать размеры статорного кольца, характеризуемые высотой $\overline{b_0}$ и диаметром входных $\overline{D_a}$ и выходных $\overline{D_b}$ статорных колонн. Высота $\overline{b_0}$ обозначена на габаритных чертежах турбин\footnote{Величина $\overline{b_0}$ определялась при выборе фактических параметров  турбин по данным таблиц \ref{tab_4}, \ref{tab_6} и \ref{tab_7}.} и зависит от их типа. Размеры статора не намного отличаются от размеров направляющего аппарата и могут быть приняты равными\footnote{Высота статора $\overline{b'_0} = \overline{b_0} + 0.0033 $ несколько больше высоты направляющего аппарата. Это обстоятельство учитывается при проектировании спиральной камеры, однако в виду малости их различия в дальнейшем будем считать $\overline{b'_0} = \overline{b_0}$. Поэтому на рисунке \ref{img_ppo} на расстоянии $R_a$ от оси турбины сразу показана величина $b_0$ (высота направляющего аппарата).}.


Для турбин с бетонными спиральными камерами относительный диаметр окружности, проведённой по входным кромкам статорных колонн, назначают в пределах:

$$
   \overline{D_a} = 1.5 - 1.55 \; .
$$ 

И величину относительного диаметра окружности, проведённой по выходным кромкам статорных колонн, принимают равной:


$$
  \overline{D_b} = 1.3 - 1.35 \; .
$$

Б$\acute{\text{о}}$льшие значения $\overline{D_a}$ и $\overline{D_b}$ принимают для турбин с диаметрами рабочих колёс $D_1~<~4.0$~м.

При известном диаметре рабочего колеса принятой турбины $D_1$ определяют величины $D_a$ и $D_b$ и соответствующие им радиусы $R_a$ и $R_b$:

$$
   D_a = \overline{D_a} \cdot D_1 \; ; \; \; R_a = \frac{D_a}{2}
$$

$$
   D_b = \overline{D_b} \cdot D_1 \; ; \; \; R_b = \frac{D_b}{2} \; .
$$




\vspace{0.5cm}

\begin{center}
Условия применения различных форм поперечного сечения \\ бетонных спиральных камер.
\end{center}


\begin{description}
\item[-] \textbf{Симметричное сечение} (Рис. \ref{img_ppo}, а) $b_1 = b_2$. Считается наиболее эффективным в гидравлическом отношении, то есть при такой форме спиральной камеры минимальны гидравлические потери. Также, благодаря развитию поперечного сечения вверх и вниз, такая форма может иметь меньшую ширину при одинаковых площадях поперечного сечения, по сравнению с остальными формами поперечных сечений. С уменьшением ширины поперечного сечения спиральной камеры также уменьшаются её размеры в плане, что позволяет более компактно размещать гидроагрегаты в здании ГЭС.

Вариант спиральной камеры с такой формой сечения следует рассматривать в первую очередь. Однако, иногда, выступающая вверх $b_1$ часть спиральной камеры мешает размещению сервомоторов направляющего аппарата при компоновке механизмов. Или же, наоборот, выступающая вниз $b_2$ часть сечения спиральной камеры препятствует возможности расположения водопропускных отверстий в здании ГЭС с донными водосбросами. В таких случаях принимают к рассмотрению другие формы поперечных сечений спиральной камеры.

\item[-] \textbf{Развитое вверх} сечение спиральной камеры (Рис. \ref{img_ppo}, б) $b_1 > b_2$. Такое сечение спиральной камеры принимают в случае необходимости размещения водосбросных каналов для пропуска паводковых вод через здание ГЭС.

\item[-] \textbf{Развитое вниз} сечение спиральной камеры (Рис. \ref{img_ppo}, в) $b_1 < b_2$. Этот тип сечения принимают для улучшения условий размещения оборудования для управления направляющим аппаратом.

\item[-] Сечения \textbf{с горизонтальным потолком} $b_1 = 0$ и \textbf{с горизонтальным полом} $b_1 = 0$. (Рис. \ref{img_ppo}, г и д) применяют в тех же случаях, что и сечения, развитые вверх или вниз, соответственно. Разница заключается лишь в большем количестве освобождаемого для достигаемых целей пространства.

\end{description}

\vspace{0.5cm}

Величины $B$ и $R$ различны для разных поперечных сечений, определяемых углом $\varphi$. Так величина $b$ по мере сужения сечения спиральной камеры изменяется (за счёт уменьшения $b_1$ и $b_2$) до величины $b_0$. При этом ширина поперечного сечения спиральной камеры, определяемая величиной $R$ уменьшается от $R_{\text{вх}}$ до $R_b$.

Характер изменения величин $b$ и $R$ устанавливается в ходе гидравлического расчёта спиральной камеры.

\vspace{0.5cm}

При напорах более 50 м бетонные спиральные камеры снабжаются стальной облицовкой из листов толщиной 10 -- 16 мм. Эта облицовка в основном служит противофильтрационной защитой, но учитывается при расчёте конструкции на прочность. Она должна быть хорошо заанкерена в бетон.

\vspace{1.5cm}

\textbf{Металлические спиральные камеры} имеют угол охвата $\varphi_{\text{охв}} = 340 - 350^\circ$. Поперечные сечения круглой формы, причём по мере перемещения от входа к концу, с уменьшением расхода площадь сечения и радиус $R$ убывают. В концевой части, примерно на последних $90^\circ$, сечения переходят в эллиптические, что объясняется тем, что необходимая для равномерного распределения расхода площадь здесь настолько мала, что круглое сечение невозможно совместить с опорными кольцами статора.

Металлическая спиральная камера представляет собой сварную конструкцию, выполненную из вальцованных стальных листов. Листы по всему периметру приварены к верхнему и нижнему опорным кольцам статора. 


\begin{figure} [ht]
  \center
  \includegraphics [scale = 0.9] {LogoAsf}
  \caption{Поперечное сечение металлической спиральной камеры.}
  \label{img_ppp}
\end{figure}


Поперечное сечение металлической спиральной камеры характеризуется следующими величинами:

$b_0$ -- высота направляющего аппарата;

$r$ -- радиус поперечного сечения;

$R$ -- величина, характеризующая ширину спиральной камеры;

$R_a$ -- радиус окружности, проведённой по входным кромкам направляющего аппарата;

$R_b$ -- радиус окружности, проведённой по выходным кромкам направляющего аппарата.

\vspace{0.5cm}

Высота направляющего аппарата $b_0$ определяется по величине относительной высоты направляющего аппарата $\overline{b_0}$, которая известна для выбранной турбины с диаметром рабочего колеса $D_1$.

$$
   b_0 = \overline{b_0} \cdot D_1
$$

Для турбин с металлическими спиральными камерами относительный диаметр окружности, проведённой по входным кромкам статорных колонн назначают:

$$
  \overline{D_a} = 1.55 - 1.64
$$

Величину диаметра окружности, проведённой по выходным кромкам статорных колонн принимают равной:

$$
  \overline{D_b} = 1.33 - 1.37
$$

Б$\acute{\text{о}}$льшие значения $\overline{D_a}$ и $\overline{D_b}$ принимают для турбин с диаметрами рабочих колёс $D_1~<~4.0$~м.


По известным таким образом величинам $\overline{D_a}$ и $\overline{D_b}$ определяют $\overline{R_a}$ и $\overline{R_b} \,$:

$$
   D_a = \overline{D_a} \cdot D_1 \; ; \; R_a = \frac{D_a}{2} \; ;
$$


$$
   D_b = \overline{D_b} \cdot D_1 \; ; \; R_b = \frac{D_b}{2} \; .
$$


Величины $r$ и $R$ уменьшаются по мере сужения спиральной камеры при повороте на угол $\varphi_{\text{охв}}$. Характер их изменения (уменьшения) устанавливают при проведении гидравлического расчёта спиральной камеры.























\newpage

Проглотил дед зайца, а заяц ему говорит: (см. Пример \ref{Primer2})  

\rotatebox{90}{ 5656565} %это обеспечивает поворот любого объекта

вфывфывф\footnote{KUM \ref{tab_6} \ref{tab_7}}

22223
\No1

 \' uuu

\begin{figure} 





  \center
  \includegraphics [scale = 0.5, angle = 30] {LogoAsf}
  \caption{Название.}
  \label{img_asdf}




\end{figure}
















