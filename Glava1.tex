\newtheorem{primer}{Пример}

\section {Общие сведения.}			% Заголовок

\subsection {Источники энергии.}

Одной из важнейших задач человечества является обеспечение потребности в электроэнергии. Мировая выработка электроэнергии составляет более 12 трлн кВт$\cdot$ч в год ($12 \cdot 10^{12}$ кВт$\cdot$ч).

Источники электроэнергии подразделяются на возобновляемые и не-возобновляемые.

К не-возобновляемым источникам энергии относятся:
\begin{description}
\item[-] Нефть и нефтепродукты (мазут, бензин, керосин, солярка и т.д.).
\item[-] Природный газ.
\item[-] Каменный уголь и торф.
\item[-] Атомная энергия (уран).
\end{description}

К возобновляемым источникам энергии (ВИЭ) относятся:
\begin{description}
\item[-] Вода.
\item[-] Солнце.
\item[-] Ветер.
\item[-] Биомасса (растения, животные).
\item[-] Геотермальные источники.
\end{description}

\vspace{0.5cm}

Приблизительное распределение различных типов энергоресурсов в общей выработке электроэнергии (World Energy Statistics 2010):
\begin{description}
\item[Нефть и нефтепродукты] --- 33 \%
\item[Природный газ] --- 21 \%
\item[Каменный уголь и торф] --- 27 \%
\item[Атом] --- 6\%
\item[Горючие ВИЭ и отходы] --- 10 \%
\item[Вода] --- 2 \%
\item[Прочие ВИЭ] --- 1 \%
\end{description}

\vspace{0.5cm}

Выработка электроэнергии осуществляется на следующих видах электростанций:
\begin{description}
\item[ТЭС] --- тепловые электростанции.
\item[АЭС] --- атомные электростанции.
\item[БЭС] --- биотопливные электростанции.
\item[ГЭС] --- гидроэлектростанции.
\item[ГАЭС] --- гидроаккумулирующие электростанции.
\item[ПЭС] --- приливные электростанции.
\item[ВЭС] --- ветряные электростанции.
\item[СЭС] --- солнечные электростанции.
\item[ГТЭС] --- геотермальные электростанции.
\end{description}


Запасы не-возобновляемых источников постоянно сокращаются. По предварительным данным их хватит приблизительно на сто лет. Наиболее обеспеченной является атомная энергия, запасов урана хватит на несколько сотен лет. Однако АЭС не могут быть использованы для выработки всей мировой энергии, так как в результате их работы остаётся большое количество отходов-долгожителей (отработанные стержни перед окончательным захоронением предварительно содержат в течение двух - трёх лет в специальных бассейнах) и возникает проблема их захоронения. 

Среди возобновляемых источников энергии наиболее обеспеченным является источник гидравлической энергии, однако существующие ГЭС не способны покрыть общую потребность в электроэнергии. 

Другие ВИЭ занимают незначительную долю в общей выработке электроэнергии. При освоении этих ВИЭ возникает ряд затруднений:
\begin{enumerate}
\item Неравномерность распределения источников. 
\item Высокая стоимость и сложность возведения станций.
\item Сложность аккумулирования энергии и передаче её на дальние расстояния.
\end{enumerate}

По данным на 2010 год лидерами в развитии производства из альтернативных источников энергии были следующие страны:
\begin{description}
\item[Исландия] --- 25 \% (в основном, геотермальная энергия).
\item[Дания] --- 20.6 \% (энергия ветра).
\item[Португалия] --- 18 \% (энергия волн, солнца, ветра).
\item[Новая Зеландия] --- 15.1 \% (геотермальная энергия, энергия ветра) \%.
\end{description}

В настоящее время проводятся исследования в области получения энергии на основе термоядерного синтеза. Это реакция слияния ядер, которая реализуется за счёт теплового движения, при этом ядра сближаются настолько, что преодолеваются силы отталкивания и начинают действовать силы притяжения между ядрами атомов. Для запуска реакции необходимо разогреть компоненты (до 100 млн $^\circ$C) и удерживать их. При этом вещество переходит в состояние плазмы. Электроны начинают отрываться от ядер, но удерживаются очень сильным магнитным полем, а нейтроны передают свою энергию стенкам реактора. Стенки реактора охлаждают (например, жидким литием). Образовавшийся пар направляют на турбину.

Экспериментальная установка для получения термоядерной энергии: ТОКАМАК --- тороидальная камера с магнитной катушкой.

Пример термоядерной реакции:

$$
 ^2_1 H(D) + ^3_1H(T) \rightarrow ^4_2 He + ^1_0 n + \text{энергия (17.6 МэВ)} ,
$$
где $^2_1 H(D)$ -- дейтерий; \\

$^3_1H(T)$ -- тритий; \\

$^4_2 He$ -- гелий 4; \\

$^1_0 n$ -- нейтрон. \\

Такое получение электроэнергии является экологически безопасным.

\vspace{1.0cm}

К водным ресурсам Земли относятся:
\begin{description}
\item[-] Воды морей и океанов.
\item[-] Воды озёр и водохранилищ.
\item[-] Воды подземных и наземных водотоков.
\item[-] Грунтовые воды.
\item[-] Почвенная влага.
\item[-] Ледники.
\end{description}

Общие мировые запасы водных ресурсов оцениваются примерно в 1500 млн км$^3$. Из них на долю озёр и водохранилищ приходится 750 км$^3$, а в реках в каждый момент времени протекает 1200 км$^3$. 

Особое значение водные ресурсы имеют для преобразования их энергии в электрическую энергию. Мировые запасы водной энергии составляют около 75 млрд кВт$\cdot$ч. Они распределены по континентам следующим образом:

\begin{description}
\item[Европа] --- 2.1 млрд кВт$\cdot$ч.
\item[Азия] --- 11.75 млрд кВт$\cdot$ч.
\item[Африка] --- 6.15 млрд кВт$\cdot$ч.
\item[Америка] --- 11.4 млрд кВт$\cdot$ч.
\item[Австралия] --- 1.5 млрд кВт$\cdot$ч.
\end{description}

Странами-лидерами по выработке гидроэнергии, приходящейся на каждого жителя являются: Исландия, Норвегия, Канада и Швеция. Крупнейшими производителями по абсолютным показателям являются:
\begin{enumerate}
\item Китай. 
\item Канада.
\item Бразилия (по данным 2008 г.).
\item США. 
\item Россия.
\end{enumerate}

Для преобразования водной энергии в электрическую возводят специальные гидротехнические сооружения -- гидроэлектростанции (ГЭС).

\newtheorem{opred}{Определение}[section]
\begin{opred}
   ГЭС -- гидротехническое сооружение, предназначенное для преобразования механической энергии потока в электрическую.
\end{opred}




%%_________________________________________________________________________________________________________________________
%%_________________________________________________________________________________________________________________________
%%_________________________________________________________________________________________________________________________

\subsection {Мощность водотока.}

Водотоки обладают большими запасами энергии, которые могут быть использованы для преобразования в электрическую энергию.

Выведем выражение для величины мощности водотока, для этого рассмотрим участок потока с расходом Q, ограниченный сечениями 1-1 и 2-2.

\begin{figure} [ht]
  \center
  \includegraphics [scale = 0.9] {pa}
  \caption{Участок рассматриваемого водотока.}
  \label{img_pa}
\end{figure}

В процессе своего движения от сечения 1-1 к сечению 2-2 поток жидкости совершает работу по преодолению сил трения, которые проявляются в виде гидравлических потерь. Эти потери численно равны разности полных удельных энергий в сечениях 1-1 и 2-2 в соответствием с уравнением Бернулли.

$$
   h_{f 1-2} = \text{Э}^{\text{полн.}}_{\text{уд.1}} - \text{Э}^{\text{полн.}}_{\text{уд.2}} = \left( \frac{V_1^2}{2g} + \frac{p_1}{\rho g} + z_1 \right) - \left( \frac{V_2^2}{2g} + \frac{p_2}{\rho g} + z_2 \right)
$$ 

Если для рассматриваемого потока создать такие условия, при которых силы трения были бы минимальны, эту разницу энергий можно было бы использовать для получения электрической энергии.

Принимая во внимание, что скорость движения потока на участке 1-2 изменяется незначительно:

$$
   V_1 \approx V_2
$$

$$
   \frac{V_1^2}{2g} \approx \frac{V_2^2}{2g}
$$

\begin{equation}
\label{eq_fa}
   \text{Э}^{\text{полн.}}_{\text{уд.1}} - \text{Э}^{\text{полн.}}_{\text{уд.2}} \approx  \left( \frac{p_1}{\rho g} + z_1 \right) - \left( \frac{p_2}{\rho g} + z_2 \right) = H
\end{equation}

То есть изменение полной удельной энергии потока равно разности гидростатических напоров, величина которых соответствует отметкам свободной поверхности воды в рассматриваемых сечениях (см. Рис.\ref{img_pa}).

Удельной энергией (напором) называют отношение механической энергии потока, измеряемой в [Дж] к единице веса рассматриваемой жидкости.

$$
   \text{Э}_{\text{уд.}} = \frac{E}{G} = \frac{E}{mg}  \quad \left[ \frac{\text{Дж}}{\text{Н}} \right] = \left[ \frac{\text{Н} \cdot \text{м}}{\text{Н}} \right] = \left[ \text{м} \right]
$$

Чтобы получить величину энергии, теряемой потоком на участке 1-2 в [Дж], выполним обратное преобразование, то есть разницу полных удельных энергий (см. ф. \ref{eq_fa}) умножим на вес жидкости. Таким образом, величина энергии потока:

$$
   E = \left( \text{Э}^{\text{полн.}}_{\text{уд.1}} - \text{Э}^{\text{полн.}}_{\text{уд.2}} \right) \cdot G \approx H \cdot G \, ,
$$ 
где $G = mg$ -- вес жидкости, перемещающейся из сечения 1-1 в сечение 2-2 [Н].

Поскольку расход потока известен, вес жидкости, проходящей за время $\Delta t$ через сечения 1-1 и 2-2 равен:

$$
   G = mg = W \rho g = Q \Delta t \rho g \, ,
$$
где $ m $ -- масса жидкости, проходящий через поперечное сечение потока;

$ W $ -- объём жидкости, проходящий через поперечное сечение потока;

$ \rho $ -- плотность рассматриваемой жидкости;

$ g $ -- ускорение свободного падения.

И, таким образом, механическая энергия потока:

$$
   E= H G = H Q \Delta t \rho g \quad \left[ \text{Дж} \right].
$$

Соответственно, мощность потока:

\begin{equation}
\label{eq_fb}
   N = \frac{E}{\Delta t} = H Q \rho g \quad \left[ \frac{\text{Дж}}{\text{c}} \right] = \left[ \text{Вт} \right].
\end{equation}

Поскольку в большинстве случаев ГЭС возводят на пресных водотоках Земли ($\rho = 1000 \left[ \frac{\text{кг}}{\text{м}^3} \right]$, $g = 9.81 \left[ \frac{\text{м}}{\text{с}^2} \right]$), часто полученную формулу упрощённо записывают:

$$
   N = 9.81\cdot H Q \quad \left[ \text{кВт} \right].
$$

\vspace{0.5cm}

В электроэнергетике для измерения количества выработанной электроэнергии часто используется внесистемная единица измерения -- киловатт-час [кВт$\cdot$ч] (в системе СИ количество энергии измеряют в Джоулях).

\begin{opred}
   Один киловатт-час равен количеству энергии, потребляемой (или производимой) устройством мощностью один киловатт в течение одного часа.
\end{opred}

$$
   1 \left[ \text{кВт} \cdot \text{ч} \right] = 1000 \left[ \text{Вт} \right] \cdot 3600 \left[ \text{с} \right] = 3,6 \left[ \text{МДж} \right].
$$


\textit{\underline{Например}}: электролампа мощностью 100 [Вт], освещая в течение 15 минут, потребит из электросети $ 0.1 \cdot 0.25 = 0.025 \left[ \text{кВт} \cdot \text{ч} \right] $.

\vspace{0.5cm}

Из выражения (\ref{eq_fb}) видно, что мощность потока зависит от двух величин:

\begin{enumerate}
  \item Падение на участке -- H.
  \item Расход воды -- Q.
\end{enumerate}

Таким образом, можно получить значительные мощности, а, следовательно, и выработку электроэнергии при больших значениях H и Q, однако такие условия встречаются редко. Чаще всего энергию вырабатывают при больших Q и малых H, либо при больших H и малых Q.

Большинство водотоков Земли протекает по равнинной местности, но на определённых участках имеются перепады в виде водопадов. Зачастую перепады высот создают искусственно, при помощи плотин.


\newpage

%%_________________________________________________________________________________________________________________________
%%_________________________________________________________________________________________________________________________
%%_________________________________________________________________________________________________________________________

\subsection {Типовые схемы ГЭС.}

\textbf{1. Плотинная схема.}

\begin{figure} [ht]
  \center
  \includegraphics [scale = 0.9] {pb}
  \caption{Плотинная схема ГЭС.}
  \label{img_pb}
\end{figure}

Плотинная схема широко используется при возведении русловых ГЭС, при этом здание ГЭС, в котором располагаются гидроагрегаты, может быть расположено различным образом.

С случае размещения здания ГЭС непосредственно в  теле плотины, ГЭС называют \textbf{\underline{плотинными}}. Плотинные ГЭС возводят при напорах 30 -- 40 м.

При размещении здания ГЭС со стороны нижнего бьефа, за плотиной, ГЭС называют \textbf{\underline{приплотинными}}. Приплотинные ГЭС сооружают при напорах более 40 м.

При относительно малых напорах до 30 м возводят \textbf{\underline{сифонные}} ГЭС. В сифонных ГЭС вода к гидроагрегатам, находящимся в теле плотины, подводится по изогнутым сифонным водоводам.

\vspace{0.5 cm}

\textbf{2. Деривационная схема.}

\textit{(лат. derivatio -- отведение)}

\begin{figure} [ht]
  \center
  \includegraphics [scale = 0.9] {pc}
  \caption{Деривационная схема ГЭС.}
  \label{img_pc}
\end{figure}

Эта схема предполагает отведение воды из естественного русла с большим уклоном в канал или тоннель с меньшим уклоном. Это позволяет при сравнительно небольшой длине деривации получать большие падения, а следовательно и большую выработку электроэнергии.

Деривация может быть эффективно использована на горных реках, где позволяет избавиться от затопления водохранилищем значительных площадей ценных плодородных земель, а также на меандрирующих реках.

\begin{figure} [ht]
  \center
  \includegraphics [scale = 0.9] {pd}
  \caption{Деривационный канал на меандрирующем русле.}
  \label{img_pd}
\end{figure}

Деривация также может быть использована в горной местности для выработки электроэнергии путём переброски части стока из одной речной долины в другую. 

Деривационный канал может быть как безнапорным, так и напорным.

\vspace{0.5 cm}

Деривационная схема состоит из двух узлов:

I узел -- \textbf{Головной}. В его состав входят:
\begin{description}
\item[-] Невысокая плотина.
\item[-] Водоприёмник.
\item[-] Водосброс для сброса ливневых и весенних паводковых вод.
\item[-] Отстойник для удаления взвешенных твёрдых частиц, которые могут вывести из строя турбину.
\item[-] Другие сооружения.
\end{description}

\vspace{0.5 cm}

II узел -- \textbf{Станционный}. В его состав входят:
\begin{description}
\item[-] Деривационный канал (тоннель).
\item[-] Напорный бассейн.
\item[-] Турбинный водовод.
\item[-] Здание ГЭС.
\item[-] Другие сооружения.
\end{description}

В ГЭС с напорной деривацией для защиты водовода от гидравлического удара возводят специальные сооружения -- уравнительные резервуары (башни).

\vspace{0.5 cm}

\textbf{3. Комбинированная схема.}

Предполагает создание перепада уровней H при помощи плотины и деривации. Иногда для увеличения напора здание ГЭС располагают под землёй.

\vspace{0.5 cm}

\textbf{4. Приливная схема.}

ГЭС, возводимые по приливной схеме называют приливные электростанции ПЭС. Для возведения ПЭС требуется обязательное наличие регулирующих ёмкостей (залива, или бассейнов).

\begin{figure} [ht]
  \center
  \includegraphics [scale = 0.9] {pe}
  \caption{Приливная схема ГЭС.}
  \label{img_pe}
\end{figure}

В данной схеме выработка электроэнергии производится с использованием перепада уровней воды H дважды:

1. Во время прилива отметка уровня воды в море выше отметки уровня воды, поддерживаемой в бухте, изолированной от моря плотиной. Поток воды направляется на турбину со стороны моря, при этом уровень воды в бухте постепенно повышается. После выравнивания уровней выработка энергии приостанавливается до наступления отлива. 	

2. Во время отлива. Поток воды направляется на турбину со стороны бухты, в которой уровень воды сохранялся неизменным со времени прекращения выработки электроэнергии в период прилива.

Такие электростанции работают циклично с периодом в одни сутки.

Возможны варианты приливной схемы, в которой изолированная бухта разделена плотинами на несколько бассейнов для увеличения продолжительности периодов выработки электроэнергии. В таких случаях выработка производится с последовательным использованием перепадов уровней воды в этих бассейнах.


\vspace{0.5 cm}

\textbf{5. Схема гидроаккумулирующей электростанции (ГАЭС).}

Такая схема предполагает создание двух бассейнов: верхнего и нижнего.

\begin{figure} [ht]
  \center
  \includegraphics [scale = 0.9] {pf}
  \caption{Схема гидроаккумулирующей ГЭС.}
  \label{img_pf}
\end{figure}

ГАЭС работает в двух режимах:

1. Турбинный режим -- вода с напором H подаётся на турбину с верхнего бассейна, производится выработка электроэнергии. Станция работает в турбинном режиме в часы пик во время недостатка электроэнергии в центральной сети.

2. Насосный режим -- вода отводится из нижнего бассейна в верхний, при этом из электросети потребляется электроэнергия. В насосном режиме станция работает часы минимального электропотребления (ночью), когда в электросети имеется  избыток электроэнергии за счёт непрерывной работы ТЭЦ и АЭС.

Таким образом, ГАЭС выравнивает график нагрузки энергосистем. При строительстве таких станций могут быть применены обратимые гидромашины (насосотурбины), которые могут работать как в насосном, так и в турбинном режимах.



%%_________________________________________________________________________________________________________________________
%%_________________________________________________________________________________________________________________________
%%_________________________________________________________________________________________________________________________

\subsection [Основные элементы конструкций ГЭС.]{Основные элементы конструкций ГЭС (на примере плотинной ГЭС).}

\begin{figure} 
  \center
  \includegraphics [scale = 1.0] {pj}
  \caption{Схема плотинной ГЭС.\\ \textit{ 1 -- Рабочее колесо; 2 -- Спиральная камера; 3 -- Герератор; 4 -- Вал турбины; 5 -- Отсасывающая труба; 6 -- Направляющий аппарат; 7 -- Статор турбины; 8 -- Паз ремонтного затвора; 9 -- Паз размещения сороудерживающих решёток; 10 -- Паз рабочего затвора; 11 -- Паз ремонтного затвора отсасывающей трубы; 12 -- Шпунт; 13 -- Мост; 14 -- Машинный зал; 15 -- Вспомогательные помещения.} }
  \label{img_pj}
\end{figure}

1. Рабочее колесо турбины. Турбина является основным элементом ГЭС. На ней механическая энергия потока преобразуется в механическую энергию вращения вала 4, за счёт воздействия потока на лопасти рабочего колеса.

2. Спиральная камера. Служит для равномерной подачи воды к турбине со всех сторон. Спиральная камера представляет собой постепенно сужающийся водовод, охватывающий по окружности направляющий аппарат 6. За счёт постепенного сужения вдоль потока происходит вытеснение части расхода в направлении рабочего колеса.

3. Генератор. В генераторе механическая энергия вращения вала турбины преобразуется в электрическую энергию. Генератор, смонтированный на одном валу с гидравлической турбиной, представляет собой единую конструкцию, называемую \textbf{\underline{гидроагрегат}}. В конструкциях ГЭС применяют синхронные генераторы переменного тока. 

4. Вал турбины. Вал передаёт механическую энергию вращения от рабочего колеса турбины к генератору. Вал может быть как цельным, так и составным. В случае составного вала вал турбины и вал генератора жёстко соединены фланцем, это соединение обеспечивает возможность раздельного демонтажа генератора и турбины, что снижает требования к грузоподъёмности кранового оборудования в машинном зале. 

5. Отсасывающая труба. Предназначена для отвода потока воды от гидроагрегата, а также для увеличения коэффициента полезного действия (к.п.д.) турбины. Поперечное сечение сечение отсасывающей трубы постепенно увеличивается вдоль пути следования потока. Это позволяет создать на выходе из рабочего колеса турбины дополнительное разрежение (понижение давления), что в целом увеличивает напор, срабатываемый гидроагрегатом\footnote{Это увеличение напора приводит к повышению к.п.д. турбины, поскольку в формуле для определения мощности напор входит как разность отметок верхнего и нижнего бьефов.}.

6. Направляющий аппарат. Состоит из направляющих лопаток, каждая из которых может вращаться вокруг вертикальной оси. Направляющий аппарат создаёт необходимое направление потока перед рабочим колесом, а также позволяет регулировать расход воды, поступающий на турбину. Лопатки направляющего аппарата установлены таким образом, что могут полностью перекрыть поток воды, поступающий на турбину из спиральной камеры. 

7. Статор турбины (Статорные колонны). Представляет собой ряд колонн обтекаемой формы. Основное назначение статора -- воспринимать нагрузки, возникающие между верхним и нижним железобетонными конусами, вызываемые весом конструкций и оборудования и давлением воды.

8. Паз ремонтного затвора. Служит для размещения затвора, устанавливаемого для ремонта спиральной камеры и гидроагрегата.

9. Паз размещения сороудерживающих решёток. Служит для размещения сороудерживающих решёток, препятствующих попадания мусора в гидроагрегат.

10. Паз рабочего затвора ГЭС. Служит для размещения рабочего затвора, который изолирует гидроагрегат, когда выработка электроэнергии на нём не производится.

11. Паз ремонтного затвора отсасывающей трубы. Служит для размещения затвора, устанавливаемого для ремонта отсасывающей трубы. 

12. Шпунт. Предназначен для увеличения длины пути фильтрационного потока и снижения величины фильтрационного противодавления, то есть для повышения устойчивости здания ГЭС на опрокидывание.

13. Мост. Устраивается для обеспечения проезда транспорта.

14. Машинный зал. В машинном зале размещаются мостовые краны, позволяющие демонтировать гидроагрегаты в процессе ремонта.

15. Вспомогательные помещения. Здесь размещается контрольно-измерительная аппаратура.

Обычно со стороны нижнего бьефа размещают трансформаторные подстанции, предназначенные для передачи электрического тока от гидроагрегатов ГЭС в сеть высокого напряжения. Для предотвращения размыва дна в верхнем и нижнем бьефах предусматривают крепление дна железобетонными плитами.






%%_________________________________________________________________________________________________________________________
%%_________________________________________________________________________________________________________________________
%%_________________________________________________________________________________________________________________________

\subsection{Мощность ГЭС.}
\label{moshnost}
Для определения мощности ГЭС представим водоток, на котором проектируется установить только один гидроагрегат, вырабатывающий электрическую энергию. Как уже было показано, мощность водотока можно определить по зависимости (\ref{eq_fb}):

$$
  N = \rho g Q H .
$$

Однако, не вся эта мощность будет передана валу турбины, так как при движении жидкости вдоль лопастей рабочего колеса, а также при вращении самого рабочего колеса возникают потери. Величина этих потерь учитывается коэффициентом полезного действия турбины:

$$
   \eta_{\text{ турб.}} = \frac{N_\text{в}}{N} \, ,
$$
где $ \eta_{\text{ турб.}} $ -- к.п.д. турбины (гидравлический к.п.д.);

$ N $ -- мощность водотока [Вт]; 

$ N_\text{в} $ -- мощность на валу турбины [Вт].

\vspace{0.5 cm}

Коэффициент полезного действия турбины показывает на сколько в процентном соотношении мощность на валу турбины меньше мощности потока, проходящего через неё.

Вал турбины гидроагрегата передаёт вращение ротору генератора электрического тока. Не вся мощность на валу турбины будет преобразована в генераторе в электрическую энергию. Часть мощности на валу при преобразовании энергии теряется. Эти потери учитываются коэффициентом полезного действия генератора:

$$
   \eta_{\text{ ген.}} = \frac{N_\text{г.а.}}{N_\text{в}} \, ,
$$
где $ \eta_{\text{ ген.}} $ -- к.п.д. генератора;

$ N_\text{г.а.} $ -- электрическая мощность гидроагрегата [Вт]. 

\vspace{0.5 cm}

Таким образом, выражение для величины мощности гидроагрегата, через характеристики водотока имеет вид:

$$
   N_\text{г.а.} = \eta_{\text{ ген.}} \cdot N_\text{в} = \eta_{\text{ ген.}} \cdot \eta_{\text{ турб.}} \cdot N 
$$
 
$$
   N_\text{г.а.} = \eta_{\text{ ген.}} \cdot \eta_{\text{ турб.}} \cdot \rho g Q H
$$

Ранее мы представляли, что водотоке установлен только один гидроагрегат. В случае установки нескольких гидроагрегатов мощность водотока будет разделена между ними. Итак, в случае установки m гидроагрегатов на водотоке, мощность водотока, приходящаяся на один гидроагрегат составит:

$$
   N_0 = \frac{N}{m} = \frac{\rho g Q H}{m}
$$ 

И мощность одного гидроагрегата с учётом потерь в турбине и генераторе:

\begin{equation}
\label{eq_fc}
   N_\text{г.а.} = \frac{\rho g Q H \cdot \eta_{\text{ ген.}} \cdot \eta_{\text{ турб.}}}{m}
\end{equation}

Мощность всей ГЭС равна, соответственно:

\begin{equation}
\label{eq_fd}
   N_\text{ГЭС} = N_\text{г.а.} \cdot m \, ,
\end{equation}
где $ m $ -- проектное количество одновременно работающих гидроагрегатов.

\vspace{0.5 cm}

Величина напора H, входящая в формулу (\ref{eq_fc}) может быть определена различными способами. Из-за этого появляются различные выражения для величины мощности (по напору нетто, по напору брутто). Для практических расчётов при определении величины H ограничиваются учётом потерь в подводящем трубопроводе $h_{\text{пот}}$ и, таким образом, за расчётный напор принимают:

$$
   H = H_{\text{ст}} - h_{\text{пот}} \, ,
$$
где $ H_{\text{ст}} $ -- статический напор -- разность отметок бьефов, [м];

$ h_{\text{пот}} $ -- величина гидравлических потерь в подводящем трубопроводе, [м].

\vspace{0.5 cm}

Мощность на валу одной турбины можно определить зная величину расхода $Q_\text{а}$, проходящего через один гидроагрегат\footnote{Эту мощность можно определить мысленно заменив водоток, на котором расположена ГЭС, несколькими (по количеству гидроагрегатов) водотоками с расходом $Q_\text{а}$ в каждом.}. Её величина определяется по выражению:

\begin{equation}
\label{eq_fd2}
   N_\text{в.} = \eta_{\text{ турб.}} \cdot \rho g Q_\text{а} H \, .
\end{equation}

%%_________________________________________________________________________________________________________________________
%%_________________________________________________________________________________________________________________________
%%_________________________________________________________________________________________________________________________

\subsection[Виды турбин.]{Виды турбин и области их применения.}

Проточный тракт турбины состоит из трёх элементов: рабочего колеса, устройств, подводящих воду к рабочему колесу, устройств, отводящих воду от рабочего колеса. Главным элементом турбины является рабочее колесо, в котором осуществляется преобразование энергии, поэтому вид турбины, в основном, определяется формой рабочего колеса.

Широко используются четыре вида турбин, каждый из которых имеет свою область применения по напорам.

\begin{table}[ht]
\caption{Области применения различных видов турбин.}
\label{tab_1}
\centering

%\begin{center}

\begin{tabular}{|c|c|c|c|}

\hline

\multicolumn{4}{|c|}{   Виды турбин}\\
\hline

\multicolumn{3}{|c|}{Реактивные} & Активные\\

\hline
% \rule{0 pt}{15 pt}
  Осевые & Диагональные & Радиально-осевые & Ковшовые\\ 
 
% \rule{0 pt}{2 pt}   
% &   &  &   \\
  
\hline
H < 70 м & H = 40 -- 200 м & h = 50 -- 700 м & H $ \geqslant $ 400 м\\
\hline

\end{tabular}

%\end{center}

\end{table}

\vspace{0.5 cm}

Представленные в таблице диапазоны применения различных типов турбин по напорам являются приближёнными, а также перекрывают друг друга. Например, при значениях напора от 50 до 70 м могут быть применены и осевые, и диагональные, и поворотно-лопастные турбины. Выбор вида турбины производится на основании технико-экономических сопоставлений с учётом конкретных условий.

Реактивные турбины осуществляют преобразование как кинетической, так и потенциальной энергий потока. Лопасти рабочих колёс этих турбин работают в сплошном напорном потоке воды, в котором давление на входе в рабочее колесо больше, чем давление на сходе с рабочего колеса.

Активные турбины осуществляют преобразование только кинетической энергии потока. Рабочее колесо таких турбин приводится во вращение безнапорным потоком -- струёй воды.


%%_________________________________________________________________________________________________________________________

\newpage

\subsubsection{Осевые турбины}

Поток жидкости, поступающий на лопасти рабочих колёс таких турбин, протекает в направлении, параллельном оси вращения рабочего колеса.

\begin{figure} [ht]
  \center
  \includegraphics [scale = 0.9] {ph}
  \caption{Рабочее колесо пропеллерной турбины.}
  \label{img_ph}
\end{figure}

Различают осевые \textbf{\underline{пропеллерные}} турбины и осевые \textbf{\underline{поворотно-лопастные}} турбины. У пропеллерных турбин лопасти неподвижно закреплены на корпусе рабочего колеса. У поворотно-лопастных турбин лопасти расположены на втулке рабочего колеса так, что они могут поворачиваться вокруг своих осей, перпендикулярных оси вала.

\begin{figure} [ht]
  \center
  \includegraphics [scale = 0.9] {pi}
  \caption{Рабочее колесо поворотно-лопастной турбины.}
  \label{img_pi}
\end{figure}

Конструкция рабочего колеса поворотно-лопастной турбины предполагает вращение вокруг вала только втулки с закреплёнными на ней лопастями. Особенностью рабочих колёс поворотно-лопастных турбин является возможность при работе на ходу поворачивать лопасти рабочего колеса,  то есть изменять угол $\varphi$ установки лопастей. Угол установки лопастей турбины выбирается в строгом соответствии с открытием направляющего аппарата с учётом действующего напора так, чтобы достигалось наиболее высокое значение к.п.д. турбины. Эта особенность поворотно-лопастных турбин обеспечивает возможность их эффективной работы в сравнительно большом диапазоне напоров и расходов.

\begin{figure} [ht]
  \center
  \includegraphics [scale = 0.9] {pk}
  \caption{Поворот лопастей осевой турбины.}
  \label{img_pk}
\end{figure}

Таким образом поворотно-лопастные  турбины имеют двойное регулирование мощности: изменением открытия направляющего аппарата и изменением угла установки лопастей рабочего колеса $\varphi$. Это даёт большие энергетические преимущества, но в то же время приводит к значительному усложнению конструкции.

С увеличением напора приходится увеличивать поверхность лопастей. Сначала этого можно достигнуть за счёт размера лопасти, но затем приходится идти на увеличение числа лопастей. 

Нагрузка, воспринимаемая лопастями очень велика. Она распределяется по всей поверхности (например, у турбины диаметром 9.3 м с четырьмя лопастями при напоре 16 м общая нагрузка на каждую лопасть составляет 1.8 -- 2.2 МН). Лопасть работает как консоль с распределённой нагрузкой. В связи с этим корневые цилиндрические сечения лопастей, расположенные ближе к втулке более толстые и крутые, а периферийные сечения -- более тонкие и пологие.

Профилирование лопастей рабочего колеса это трудоёмкая процедура определения формы рабочего с учётом его прочностных характеристик и параметров потока жидкости.



Для поворотно-лопастных турбин существенное значение имеет размер зазоров между концами лопастей рабочего колеса и камеры. Чем меньше зазор, тем меньше протечка, тем выше к.п.д. Обычно, считается допустимым размер зазора:

$$
   \delta = 0.001 \cdot D_1 \, \, ,
$$
где $ D_1 $ -- диаметр рабочего колеса.

\vspace{0.5 cm}

Чтобы зазор сохранялся постоянным при изменении угла установки лопастей камера рабочего колеса должна быть сферической. Однако, в этом случае возникнут трудности с установкой рабочего колеса на место. Поэтому часть камеры выше оси поворота лопастей делается цилиндрической. В нижней, выходной части камере придают сферическую форму, но не по всей высоте, а таким образом, чтобы минимальный выходной диаметр -- горловина был не менее (0.98 -- 0.96)$ \cdot D_1 $, поскольку большее стеснение вызывает падение мощности и к.п.д.

\begin{figure} [ht]
  \center
  \includegraphics [scale = 0.9] {pl}
  \caption{Размещение поворотно-лопастной турбины в камере рабочего колеса.}
  \label{img_pl}
\end{figure}

Осевые турбины могут быть расположены в здании ГЭС вертикально и горизонтально. В случае горизонтального расположения турбина размещается в специальной герметичной капсуле.

Поворотно-лопастные турбины также называют турбинами Каплана.





%%_________________________________________________________________________________________________________________________


\subsubsection{Диагональные турбины}

Основное отличие диагональных турбин от осевых состоит в форме и конструкции рабочего колеса. Лопасти рабочего колеса диагональных турбин укреплены в конусообразной втулке под углом $ \theta $ к оси вращения колеса.

\begin{figure} [ht]
  \center
  \includegraphics [scale = 0.8] {pm}
  \caption{Рабочее колесо диагональной турбины.}
  \label{img_pm}
\end{figure}

Лопасти диагональных турбин также могут вращаться вокруг своих осей. Такие турбины могут иметь не только различное число лопастей, но и различный угол $ \theta $. С ростом напора величина угла $ \theta $ уменьшается. Примерная зависимость этой величины от напора:\\
при H = 40 -- 80 м \hspace{0.5 cm} $ \theta = 60^{\circ} $;\\
при H = 60 -- 130 м \hspace{0.5 cm} $ \theta = 45^{\circ} $;\\
при H = 120 -- 200 м \hspace{0.5 cm} $ \theta = 30^{\circ} $.\\


%%_________________________________________________________________________________________________________________________


\subsubsection[Радиально-осевые турбины.]{Радиально-осевые турбины. (Турбины Френсиса.)}

В проточной части рабочего колеса радиально-осевой турбины поток жидкости изменяет своё направление с радиального, параллельного радиусу рабочего колеса, на осевое, параллельное оси вращения рабочего колеса.

\begin{figure} [ht]
  \center
  \includegraphics [scale = 0.9] {pn}
  \caption{Схема рабочего колеса радиально-осевой турбины.}
  \label{img_pn}
\end{figure}

Лопасти рабочего колеса радиально-осевой турбины жёстко закреплены на верхнем и нижнем ободах. Рабочее колесо является цельным, неразъёмным\footnote{Если такие рабочие колёса приходится транспортировать по железной дороге, то их делают разъёмными и собирают на месте. Так рабочие колёса Братской ГЭС $ D_1 = 5.5$м были изготовлены на заводе в виде двух половин и на месте обе половины были сварены, что потребовало дополнительной обработки.}. При работе турбины в потоке перед рабочим колесом давление высокое, а за колесом низкое. Под действием этого перепада давления часть воды может бесполезно протекать через зазоры между вращающимися и неподвижными элементами. С целью уменьшения этих протечек применяются специальные уплотнения. 

\begin{figure} [htt]
  \center
  \includegraphics [scale = 0.7] {interb}
  \caption{Внешний вид рабочего колеса радиально-осевой турбины.\textit{(По сравнению в изображением на предыдущем рисунке (Рис.\ref{img_pn}) на данном виде рабочее колесо перевёрнуто, то есть обод рабочего колеса с диаметром $D_2$ обращён вверх.)}}
  \label{img_interb}
\end{figure} 

\begin{figure} [ht]
  \center
  \includegraphics [scale = 0.9] {po}
  \caption{Размещение радиально-осевой турбины.}
  \label{img_po}
\end{figure}

\newpage
%%_________________________________________________________________________________________________________________________


\subsubsection[Ковшовые турбины.]{Ковшовые турбины. (Турбины Пельтона.)}

Ковшовые турбины относятся к классу активных турбин. Они обязаны своим названием ковшеобразной форме лопастей рабочего колеса, которые поочерёдно подвергаются воздействию свободной струи, выходящей из сопла.

\begin{figure}
  \center
  \includegraphics [scale = 0.9] {pp}
  \caption{Принципиальная схема ковшовой турбины. \\ \textit{1 -- Рабочее колесо; 2 -- Вал турбины; 3 -- Лопасти; 4 -- Сопло; 5 -- Подводящий трубопровод; 6 -- Регулирующая игла.}}
  \label{img_pp}
\end{figure}

Сопло и рабочее колесо устанавливаются выше уровня воды, так что рабочее колесо вращается в воздухе. Мощность ковшовой турбины регулируется за счёт изменения расхода воды. Для этого служит регулирующая игла. Игла может полностью перекрыть сопло, по мере удаления иглы от сопла увеличивается проходное сечение, а следовательно, и расход.

Рабочее колесо состоит из диска с рабочими лопастями. Общее число лопастей 12 -- 40. Каждая лопасть состоит из двух криволинейных поверхностей, разделённых ножом.

\begin{figure} 
  \center
  \includegraphics [scale = 0.9] {pq}
  \caption{Лопасть рабочего колеса ковшовой турбины.}
  \label{img_pq}
\end{figure}

При натекании на лопасть струя делится ножом на две равные части и каждая обтекает криволинейную поверхность. За счёт изменеия как скорости движения воды, так и её направления создаётся давление на лопасти и образуется момент рабочего колеса, вращающий его вместе с валом.

Чтобы устранить при вращении рабочего колеса удар тыльной стороны лопасти о струю, в лопасти предусматривается специальная прорезь. 

\vspace{1.5 cm}

\newpage

Геометрические размеры турбин характеризуются расчётным диаметром рабочего колеса -- $ D_1 $. Диаметр рабочего колеса для разных типов турбин определяется по разному (см. рис. \ref{img_pi}, \ref{img_pm}, \ref{img_pn} и \ref{img_pp}).

Геометрически подобные турбины разных размеров образуют турбинную серию.

Разнообразие форм рабочих колёс гидравлических турбин связано с условиями их работы. С увеличением напора поток жидкости обладает всё большей энергией, которую необходимо ``отбирать'' от него. Для преобразования более высоких энергий рабочее колесо должно интенсивнее взаимодействовать с потоком. Поэтому для небольших напоров достаточно осевых турбин, а с увеличением напора поток ``заставляют'' изменять своё направление в проточных частях диагональных турбин и с дальнейшим увеличением напора -- в радиально-осевых турбинах.

\begin{figure} [htt]
  \center
  \includegraphics [scale = 0.4] {intera}
  \caption{Монтаж поворотно-лопастной турбины в турбинной  камере.}
  \label{img_pinter1}
\end{figure}


%%_________________________________________________________________________________________________________________________
%%_________________________________________________________________________________________________________________________
%%_________________________________________________________________________________________________________________________

\section{Основы теории рабочего процесса турбин.}
\label{a}

При исследовании движения жидкости через проточную часть гидроагрегата это движение рассматривается как сложное, состоящее из относительного и переносного.

Относительное движение -- движение, которое фиксируется наблюдателем, находящимся на рабочем колесе. (Картина такого движения может получена с видеокамеры, вращающейся вместе с рабочим колесом турбины).

Переносное движение -- это движение рабочего колеса и его лопастей. В турбинах это всегда вращение.

Абсолютное движение -- сумма относительного и переносного движений.

$$
   \vec{V} = \vec{U} + \vec{W} \, ,
$$
где $ V $ -- скорость абсолютного движения;

$ U $ -- скорость переносного движения;

$ W $ -- скорость относительного движения.

\vspace{1.0 cm}

\begin{center}
\textbf{Вращательное движение.}
\end{center}

Рассмотрим движение точки А по окружности радиусом r.

\begin{figure} [ht]
  \center
  \includegraphics [scale = 0.9] {pvr}
  \caption{Вращательное движение.}
  \label{img_pvr}
\end{figure}

Такое движение характеризуется угловой скоростью $\omega$. При равномерном вращении величина угловой скорости может быть определена по выражению:

$$
  \omega = \frac{\varphi}{\Delta t} \quad , \left[ \frac{\text{рад}}{\text{с}} \right]
$$
где $\varphi$ -- величина угла поворота точки А за промежуток времени $\Delta t$ в радианах [рад];

$\Delta t$ -- рассматриваемый промежуток времени [с].

\vspace{0.5 cm}

В случае полного оборота: $ \varphi_{\text{полн.}} = 2 \pi $ [рад].
Время, за которое точка А совершит полный оборот вокруг центра окружности называется периодом T.

$$
   T = \frac{\varphi_{\text{полн.}}}{\omega} = \frac{2 \pi}{\omega} \quad ,[\text{сек}]
$$

Число оборотов, которое совершит точка А за время t при равномерном вращении:

$$
   n_{\text{об.}} = \frac{t}{T} \quad ,[\text{оборотов}]
$$

Число оборотов, совершаемое точкой А за одну секунду, называется частотой вращения:

$$
   n = \frac{1}{T} = \frac{\omega}{2 \pi} \quad ,\left[ \frac{\text{оборотов}}{\text{сек}} \right]
$$

Линейная скорость U движения точки А может быть определена как отношение расстояния $ S_{\varphi} $, на которое переместится точка А за некоторый промежуток времени $\Delta t$, к величине промежутка времени. В случае полного оборота это расстояние равно длине окружности, а промежуток времени равен периоду T.

$$
   U = \frac{S_{\varphi}}{\Delta t} = \frac{2 \pi r}{T} = 2 \pi r \cdot n  \quad , \left[ \frac{\text{м}}{\text{сек}} \right]
$$ 

И с учётом выражения для величины периода вращения:

$$
   \displaystyle U = \frac{ 2 \pi r}{ \frac{2 \pi}{\omega} } = \omega \cdot r  \quad , \left[ \frac{\text{м}}{\text{сек}} \right] \, .
$$





\newpage

%%_________________________________________________________________________________________________________________________
%%_________________________________________________________________________________________________________________________
%%_________________________________________________________________________________________________________________________

\subsection{Поток, формируемый направляющим аппаратом.}

Поток жидкости, поступающий в рабочее колесо по пути своего следования проходит через спиральную камеру, и направляющий аппарат, минуя статорные колонны. При выборе характеристик турбин важно знать как изменяются при этом кинематические параметры потока (величина скорости и её направление).

\begin{figure} [ht]
  \center
  \includegraphics [scale = 0.9] {pr}
  \caption{Рабочее колесо в спиральной камере. \\ \textit{ 1 -- Колонны статора; 2 -- Лопатки направляющего аппарата; 3 -- Предколёсное пространство; 4 -- Рабочее колесо турбины.} } 
  \label{img_pr}
\end{figure}

\begin{figure} 
  \center
  \includegraphics [scale = 0.9] {ps}
  \caption{Скорость потока на выходе из направляющего аппарата.}
  \label{img_ps}
\end{figure}

На выходе из направляющего аппарата (окружности диаметром $ D_{02} $) жидкость движется со скоростью $ V_0 $, при этом направление движения близко к направлению выходных кромок лопаток (это направление определяется величиной угла $ \alpha_{\text{к}} $ -- углом, измеряемым между осью лопаток и касательной к окружности, проведённой по выходным кромкам лопаток направляющего аппарата, диаметром $ D_{02} $).


$$
   \alpha_0 = \alpha_{\text{к}}  
$$

Скорость движения жидкости на выходе из направляющего аппарата $ V_0 $ можно разложить на две составляющие: радиальную $ V_{0r} $ и окружную $ V_{0u} $.

\textbf{Радиальная составляющая}, направленная вдоль радиуса (к точке 0):

\begin{equation}
\label{eq_fe}
   V_{0r} = V_0 \cdot \sin\alpha_0
\end{equation}

Величина этой составляющей скорости зависит от величины расхода $ Q_{\text{а}} $, поступающего к рабочему колесу,  её можно определить по зависимости:

\begin{equation}
\label{eq_ff}
   V_{0r} = \frac{Q_{\text{а}}}{\omega} = \frac{Q_{\text{а}}}{\pi D_{02} b_0} \, ,
\end{equation}
где $ \omega = \pi D_{02} b_0 $ -- площадь живого сечения на выходе из направляющего аппарата;

$ b_0 $ -- высота направляющего аппарата.

\vspace{0.5 cm}

\textbf{Окружная составляющая}, направленная по касательной к окружности диаметром $ D_{02} $:

\begin{equation}
\label{eq_fj}
   V_{0u} = V_0 \cdot  \cos\alpha_0
\end{equation}

Величину скорости $ V_0 $, таким образом, можно определить через величину расхода по зависимостям (\ref{eq_fe}) и (\ref{eq_ff}).

$$
   V_0 = \frac{V_{0r}}{\sin\alpha_0}
$$

$$
   V_0 = \frac{Q_{а}}{\pi D_{02} b_0 \sin(\alpha_0)}
$$

Упрощённо говоря, если бы направляющего аппарата и спиральной камеры не было, величина окружной составляющей скорости была бы равна нулю (см. рис.\ref{img_pt}). Наличие ненулевой окружной скорости означает, что поток ``закручен'' относительно точки 0.

Степень ``закрученности'' потока характеризуется циркуляцией Г.

\begin{opred}
   Циркуляция -- (понятие гидромеханики) мера вихревого движения, выраженная не через угловую, а через поступательную скорость движения потока.
\end{opred}

$$
   \text{Г} = \oint\limits_{L} \vec{V} \cdot \vec{dL} \, , 
$$
где $ L $ -- замкнутый контур, проведённый в потоке жидкости;

$ \vec{dL} $ -- элементарный вектор, бесконечно малая часть контура $ L $.

\vspace{0.5 cm}

Величина циркуляции в скалярной формулировке:

$$
   \text{Г} = \oint\limits_{L} {V} \cdot \cos(\angle V dL) \cdot {dL} \, , 
$$
где $ \angle V dL $ -- угол между направлением вектора скорости $\vec{V}$ и направлением элементом контура $dL$.

\vspace{0.5 cm}

Например, определим величину циркуляции потока жидкости, набегающего под прямым углом на замкнутый контур в форме окружности. Такой поток может наблюдаться при истечении жидкости через трубчатый водосброс.

\begin{figure} [ht]
  \center
  \includegraphics [scale = 0.9] {pt}
  \caption{Поток в трубчатом водосбросе (окружная составляющая скорости равна нулю).}
  \label{img_pt}
\end{figure}

$$
   \angle V dL = 90^\circ 
$$

$$
   \cos(\angle V dL) = 0 \, \, , \text{следовательно, } \text{Г} = 0  .
$$

Определим величину циркуляции для потока жидкости, набегающего на контур L в форме окружности диаметром D со скоростью V под углом $\alpha$ к касательной в любой точке этого контура. В соответствии с зависимостью (\ref{eq_fj}) величина окружной составляющей скорости такого потока отлична от нуля.

\begin{figure} [ht]
  \center
  \includegraphics [scale = 0.8] {px}
  \caption{К определению циркуляции потока на выходе из направляющего аппарата.}
  \label{img_px}
\end{figure}

В рассматриваемом потоке $ \angle \alpha = \angle V dL = const $ во всех точках контура. Так как контур L -- окружность с длиной $ L = \pi \cdot D $, то:

$$
   \text{Г} = \oint\limits_{L} {V} \cdot \cos \alpha \cdot {dL} = L \cdot V \cdot \cos \alpha = \pi \cdot D \cdot V \cdot \cos \alpha
$$

Такая же картина наблюдается и в потоке на выходе из направляющего аппарата (где $ \alpha = \alpha_0 $), тогда циркуляция, создаваемая направляющим аппаратом (с учётом выражения (\ref{eq_fj})) может быть определена по зависимости:

\begin{equation}
\label{eq_fh}
   \text{Г}_{\text{н.а.}} = \pi \cdot D_{02} \cdot V_0 \cdot \cos \alpha_0 = \pi \cdot D_{02} \cdot V_{0u} 
\end{equation}

То есть направляющий аппарат создаёт в потоке циркуляцию $\text{Г}_{\text{н.а.}}$


\vspace{0.5 cm}

%%_________________________________________________________________________________________________________________________
%%_________________________________________________________________________________________________________________________
%%_________________________________________________________________________________________________________________________

\subsection[Закон постоянства момента скорости.]{Закон постоянства момента скорости. (Поток в предколёсном пространстве.)}
\label{posto}

В предколёсном пространстве -- между выходными кромками лопаток направляющего аппарата и входными кромками рабочего колеса жидкость движется свободно.

\begin{figure} [ht]
  \center
  \includegraphics [scale = 0.9] {py}
  \caption{Движение жидкости в предколёсном пространстве.}
  \label{img_py}
\end{figure}

Установим как изменяется скорость жидкой частицы в предколёсном пространстве при её свободном движении. Для этого используем закон механики о моменте количества движения.

\begin{opred}
   Мгновенная скорость изменения момента количества движения системы материальных точек относительно некоторой оси равна сумме моментов всех внешних сил, действующих на эту систему.
\end{opred}
(Количество движения -- $ m \cdot V$, где $ m $ -- масса; $V$ -- скорость.)

Момент количества движения -- произведение количества движения на плечо. В нашем случае при движении по окружности вокруг оси 0, плечо -- это радиус окружности r. Таким образом, момент количества движения рассматриваемой частицы:

$$
   \text{МКД} = (m \cdot V_u \cdot r)
$$

\begin{figure} [ht]
  \center
  \includegraphics [scale = 0.9] {pz}
  \caption{К определению момента количества движения.}
  \label{img_pz}
\end{figure}

Радиальная составляющая скорости $V_r$ момента количества движения не создаёт, так как направлена к центру окружности, и, соответственно, имеет плечо, равное нулю (см. рис.\ref{img_pz}).

Итак, теорема об изменении момента количества движения может быть записана в виде выражения:

$$
   \frac{d \text{МКД}}{dt} = \sum M_0 \, , \, \text{или}
$$

$$
   \frac{d(m \cdot V_u \cdot r)_0}{dt} = \sum M_0 \,,
$$
где $M_0$ -- сумма моментов сил, действующих на жидкую частицу, относительно точки 0.

\vspace{0.5 cm}
 
Индекс 0 в этом выражении указывает на то, что моменты определяются относительно точки 0.

Рассмотрим силы, действующие на жидкую частицу (см. Рис.(\ref{img_py}).

$ P_r$ и $ P_r + dP $ -- Силы давления, имеющие радиальные направления. Эти силы момента относительно точки 0 не образуют.

$ P_u $ -- Силы давления на торцевые поверхности жидкой частицы. В условиях осреднённого потока равны на противоположных гранях частицы и взаимно компенсируются.

$ \tau $ -- Касательные напряжения сил трения. При выводе закона постоянства момента скорости силами трения пренебрегают ввиду их малости. Это пренебрежение можно считать допущением для выполнения выводимого закона.

Итак, силы, действующие на жидкую частицу, свободно движущуюся в предколёсном пространстве момента относительно точки 0 не образуют, или его величиной можно пренебречь. Тогда:

$$
   \frac{d(m \cdot V_u \cdot r)_0}{dt} = 0 \,  \text{ и, следовательно,}
$$

$$
   m \cdot V_u \cdot r = const \, , \text{ так как } m = const
$$

\begin{equation}
\label{eq_fi}
   V_u \cdot r = const
\end{equation}

Полученное выражение (\ref{eq_fi}) и представляет собой закон постоянства моментов скорости, который характеризует условия свободного движения жидкости в рассматриваемом нами потоке. Из этого закона следует, что по мере приближения к оси 0 окружная составляющая скорости возрастает.

Рассмотрим как влияет такое изменение скорости на величину циркуляции. Для этого осуществим ряд преобразований выражения (\ref{eq_fh}):

$$
   \text{Г}_{\text{н.а.}} = \pi \cdot D_{02} \cdot V_{0u} = 2 \pi \cdot r_{02} \cdot V_{0u}\, \, ,
$$
где $r_{02} = \frac{D_{02}}{2}$.

В соответствии с выражением (\ref{eq_fi}):

$$
   r_{02} \cdot V_{0u} = r \cdot V_u = const \, ,
$$

таким образом,

$$
   \text{Г}_{\text{н.а.}} = \text{Г} = 2 \pi \cdot r \cdot V_u 
$$

То есть в пространстве между направляющим аппаратом и рабочим колесом циркуляция, созданная направляющим аппаратом, сохраняется.



%%_________________________________________________________________________________________________________________________
%%_________________________________________________________________________________________________________________________
%%_________________________________________________________________________________________________________________________

\subsection[Основное энергетическое уравнение турбины]{Основное энергетическое уравнение турбины. (Поток в проточной части рабочего колеса.)}

В проточной части рабочего колеса (то есть в пространстве между лопастями рабочего колеса) закон постоянства момента скорости не может выполняться, поскольку поток ограничен лопастями, которые вращаются с одинаковой угловой скоростью.
Направление потока, характеризуемое углом $\alpha$ определяется очертаниями лопастей рабочего колеса. Направление и скорость потока изменяются вдоль лопасти рабочего колеса. Чтобы охарактеризовать такой поток производят его осреднение. Так, например, в некотором горизонтальном сечении n-n рабочего колеса вертикальной осевой турбины\footnote{Последующее утверждение справедливо и для остальных видов реактивных турбин. Здесь говорится о вертикальной осевой турбине лишь в качестве выбранного примера.} осреднённый поток жидкости характеризуется следующими величинами (кинематическими параметрами):

\begin{figure} [ht]
  \center
  \includegraphics [scale = 0.9] {ppa}
  \caption{Кинематические параметры осреднённого потока.}
  \label{img_ppa}
\end{figure}

$ V_n $ -- средняя скорость потока в горизонтальном сечении n-n;

$ V_{nr} $ -- радиальная составляющая средней скорости потока в сечении n-n;

$ V_{nu} $ -- окружная составляющая средней скорости в сечении n-n;

$ \alpha_n $ -- угол, характеризующий направление средней скорости (угол между вектором средней скорости и вектором окружной составляющей средней скорости);

$ r_n $ -- радиус, характеризующий местоположение вектора средней скорости относительно оси вращения (Поскольку при переходе от одного сечения к другому форма эпюры скорости изменяется, изменяться будет и положение точки приложения вектора средней скорости, которым характеризуется эта эпюра скорости). 

\vspace{1 cm}

Далее будем рассматривать относительное движение (см. гл.\ref{a}).

Для вывода основного энергетического уравнения турбины рассмотрим рабочее колесо осевой  турбины в камере рабочего колеса.

\begin{figure} [ht]
  \center
  \includegraphics [scale = 0.9] {ppb}
  \caption{К выводу основного энергетического уравнения.}
  \label{img_ppb}
\end{figure}

Сечениями 1-1 и 2-2 выделим некоторый объём жидкости, протекающий в каждый момент времени между лопастями рабочего колеса. Боковыми ограничивающими поверхностями этого объёма являются стены камеры рабочего колеса. Движение жидкости будем считать установившимся, то есть не изменяющимся во времени. 

При прохождении потока жидкости между сечениями 1-1 и 2-2 он, взаимодействуя с лопастями рабочего колеса, теряет часть своей энергии, при этом изменяются величина и направление скорости осреднённого потока.

Применим к рассматриваемому объёму жидкости закон об изменении момента количества движения.

\begin{equation}
\label{eq_fk}
   \frac{m \cdot V_u \cdot r}{dt} = \sum M \, ,
\end{equation}
где $ m $ -- масса рассматриваемого объёма;
 
$ V_u $ -- окружная составляющая средней скорости;

$ r $ -- радиус, характеризующий местоположения вектора средней скорости;

$ \sum M $ -- сумма моментов внешних сил, действующих на жидкость.

\vspace{0.5 cm}

Поскольку движение установившееся, скорости $ V_u $  не изменяются во времени, изменения происходят лишь в пространстве: при перемещении от одного сечения к другому.

Перепишем выражение (\ref{eq_fk}) в конечных разностях для сечений 1-1 и 2-2.

\begin{equation}
\label{eq_fl}
   \frac{m V_{2u} r_2 - m V_{1u} r_1}{\Delta t} = \sum M
\end{equation}

Масса жидкости, проходящей за промежуток времени $ \Delta t $ между рассматриваемыми сечениями, зависит от величины расхода через гидроагрегат $ Q_{\text{а}} $.

\begin{equation}
\label{eq_fm}
   m = \rho \cdot Q_{\text{а}} \cdot \Delta t
\end{equation}

При движении рассматриваемого потока внутри него действуют различные силы: объёмные силы, силы давления и трения. В конечном счёте действие всех этих сил приводит к появлению момента $ M_{\text{в}} $ на валу рабочего колеса. Таким образом, можно написать:

\begin{equation}
\label{eq_fn}
   \sum M = - M_{\text{в}}
\end{equation}

Знак минус, обусловлен тем, что энергия потока уменьшается, при этом энергия, сообщаемая валу рабочего колеса возрастает. Здесь пока не учитываются потери энергии, возникающие в подшипниках вала\footnote{Эти потери учитываются коэффициентом полезного действия $ \eta_{\text{ турб}} $.} и считается, что вся энергия, сообщённая потоком лопастям рабочего колеса, передаётся на вал.

Итак, выражение (\ref{eq_fl}) с учётом (\ref{eq_fm}) и (\ref{eq_fn}) примет следующий вид:

$$
   \rho  Q_{\text{а}} \Delta t \cdot \frac{V_{2u} r_2 - V_{1u} r_1}{\Delta t} = - M_{\text{в}}
$$

И после преобразований получаем выражение для определения величины момента на валу рабочего колеса:

$$
   M_{\text{в}} = \rho  Q_{\text{а}} \cdot ( V_{1u} r_1 - V_{2u} r_2 )
$$

Окружные составляющие средней скорости в полученном выражении могут быть представлены через величины средней скорости осреднённого потока:

$$
   V_{1u} = V_1 \cdot \cos \alpha_1
$$

$$
   V_{2u} = V_2 \cdot \cos \alpha_2
$$

И тогда выражение для величины момента на валу рабочего колеса примет вид:

\begin{equation}
\label{eq_fn2}
    M_{\text{в}} = \rho  Q_{\text{а}} \cdot ( V_1 \cdot \cos \alpha_1 \cdot r_1 - V_2 \cdot \cos \alpha_2 \cdot r_2 ) 
\end{equation}

Можно также получить выражение для определения величины момента на валу рабочего колеса через изменение циркуляции. Величина циркуляции осреднённого потока в некотором сечении i может быть определена (см. ф.(\ref{eq_fh})) по зависимости:

$$
   \text{Г}_i = \pi \cdot D_i \cdot V_i \cdot \cos \alpha_i = 2 \pi \cdot r_i \cdot V_i \cdot \cos \alpha_i
$$

откуда:

\begin{equation}
\label{eq_ft}
    V_i \cdot \cos \alpha_i \cdot r_i = \frac{\text{Г}_i}{2 \pi}
\end{equation}


И величина момента на валу:

$$
    M_{\text{в}} = \frac{\rho  Q_{\text{а}} }{2 \pi} (\text{Г}_1 - \text{Г}_2)
$$

Последняя полученная формула особенно наглядна. Она показывает, что {\color{red}\textit{на рабочем колесе возникает крутящий момент только в том случае, когда оно воздействием своих лопастей изменяет циркуляцию потока}}. 

\vspace{1 cm}

Ранее в главе (\ref{moshnost}) определялась величина мощности на валу турбины через параметры водотока. Для одного гидроагрегата мощность на валу:

\begin{equation}
\label{eq_fo}
    N_{\text{в}} = N \cdot \eta_{\text{турб.}} = \rho g Q_{\text{а}} H \eta_{\text{турб.}}
\end{equation}

Мощность на валу турбины также можно определить зная величину крутящего момента на валу и величину угловой скорости вращения рабочего колеса:

\begin{equation}
\label{eq_fp}
    N_{\text{в}} = M_{\text{в}} \cdot \omega
\end{equation}
где $ N_{\text{в}} $ -- мощность на валу [Вт];

$ M_{\text{в}} $ -- крутящий момент на валу турбины [$\text{Н} \cdot \text{м}$];

$ \omega $ -- угловая скорость вращения рабочего колеса [$ \frac{1}{\text{сек}} $].

\vspace{0.5 cm}

В соответствии с этими выражениями (\ref{eq_fo}) и (\ref{eq_fp}):

\begin{equation}
\label{eq_fq}
    M_{\text{в}} = \frac{N_{\text{в}}}{\omega} = \frac{\rho g Q_{\text{а}} H \eta_{\text{турб.}}}{\omega} 
\end{equation}

Итак, имеются два выражения по которым можно определить величину крутящего момента на валу турбины. Выражение (\ref{eq_fn2}), в котором момент $ M_{\text{в}} $ определяется по параметрам осреднённого потока в проточной части рабочего колеса и выражение (\ref{eq_fq}), связывающее $ M_{\text{в}} $ с величинами напора и расхода одного гидроагрегата. Поскольку в обоих выражениях $ M_{\text{в}} $ является одной и той же величиной, можно получить выражение, устанавливающее связь между параметрами осреднённого потока в турбине, напором, расходом и угловой скоростью вращения рабочего колеса турбины.

$$
   \frac{\rho g Q_{\text{а}} H \eta_{\text{турб.}}}{\omega} = \rho  Q_{\text{а}}  ( V_1  \cos \alpha_1  r_1 - V_2  \cos \alpha_2  r_2)
$$

откуда

\begin{equation}
\label{eq_fr}
   H \eta_{\text{турб.}} = \frac{\omega}{g} ( V_1  \cos \alpha_1  r_1 - V_2  \cos \alpha_2  r_2 )
\end{equation}

Произведение величины угловой скорости на радиус представляет собой величину линейной скорости $U$ точки, находящейся на расстоянии $r$ от оси вращения.

$$ 
   U = \omega \cdot r
$$ 

С учётом этого выражение (\ref{eq_fr}) примет вид:

\begin{equation}
\label{eq_fs}
  \boxed{ H \eta_{\text{турб.}} = \frac{1}{g} \left( U_1 V_1  \cos \alpha_1 - U_2 V_2  \cos \alpha_2 \right) }
\end{equation}
где $H$ -- действующий напор.

$\eta_{\text{турб.}}$ -- к.п.д. турбины (гидравлический к.п.д.).

$g$ -- ускорение свободного падения;

$U_i$ -- линейная скорость вращения точки, находящейся на расстоянии $r_i$ от оси вращения рабочего колеса, характеризующем местоположение вектора средней скорости потока жидкости. Эта линейная скорость зависит от частоты вращения рабочего колеса;

$V_i$ -- относительная\footnote{Под относительной скоростью здесь имеется в виду скорость потока жидкости относительно скорости движения рабочего колеса.} средняя скорость потока жидкости в проточной части рабочего колеса;

$\alpha_i$ -- угол, характеризующий направление средней скорости. Величина этого угла зависит от кривизны лопастей рабочего колеса.

\vspace{0.5 cm}

Выражение (\ref{eq_fs}) представляет собой основное уравнение турбин, или уравнение Эйлера.

Можно получить другую форму основного уравнения турбин из выражений (\ref{eq_fr}) и (\ref{eq_ft}):

\begin{equation}
\label{eq_fv}
  \boxed{ H \eta_{\text{турб.}} = \frac{\omega}{g 2 \pi} \left( \text{Г}_1 - \text{Г}_2 \right) }
\end{equation}

Левая часть основного уравнения турбин представляет собой величину удельной энергии жидкости [м вод. ст.]
, сообщаемой рабочему колесу с учётом потерь энергии. Правая часть содержит кинематические параметры потока при входе в рабочее колесо и при выходе из него. Таким образом, основное уравнение даёт связь между энергетическими и кинематическими параметрами в турбине.

Из уравнения Эйлера в форме (\ref{eq_fv}) можно сделать важные выводы:

\begin{description}
\item[1.] Наиболее благоприятный для выработки энергии режим работы наблюдается когда циркуляция на выходе из рабочего колеса $\text{Г}_2 = 0$, или мала. Поскольку величина циркуляции на входе в рабочее колесо равна циркуляции, создаваемой направляющим аппаратом $\text{Г}_1 = \text{Г}_{\text{н.а.}}$ (см. гл. \ref{posto}) можно определить требуемое значение $ \text{Г}_{\text{н.а.}} $

\item[2.] В процессе прохождения воды через рабочее колесо турбины циркуляция потока должна убывать. Следовательно рабочее колесо ``срабатывает'' циркуляцию, созданную направляющим аппаратом.
\end{description}



%%_________________________________________________________________________________________________________________________
%%_________________________________________________________________________________________________________________________
%%_________________________________________________________________________________________________________________________

\subsection[Отвод жидкости от турбины]{Отвод жидкости от турбины. Высота отсасывания.}

Вода из рабочего колеса реактивных турбин выпускается в отсасывающую трубу и по ней отводится в нижний бьеф. Отсасывающая труба существенно влияет на энергетические показатели турбин, особенно низконапорных.

\subsubsection{Работа отсасывающей трубы.}

\begin{figure} [ht]
  \center
  \includegraphics [scale = 0.9] {ppc}  \caption{Гидравлические показатели отсасывающей трубы.}
  \label{ppc}
\end{figure}

На рисунке \ref{ppc} показана величина $H_s$, которая называется высотой отсасывания. Это расстояние, которое отсчитывается от определённой плоскости до отметки уровня воды в нижнем бьефе.
Высота отсасывания определяет расположение турбины по высоте, относительно нижнего бьефа. Её величина зависит от типа и размера турбины. Теория, в соответствии с которой выбирается эта величина, будет рассмотрена позже.


На выходе из рабочего колеса в сечении 2-2 поток обладает средней скоростью $V_2$ и давлением $p_2$. Величина удельной энергии потока в сечении 2-2 относительно плоскости сравнения 0-0, расположенной на уровне нижнего бьефа, равна:

\begin{equation}
\label{eq_fx}
   e_2 = \frac{p_2}{\rho g} + z_2 + \frac{\alpha_2 V_2^2 }{2 g}
\end{equation}

Энергия $e_2$, с которой вода покидает рабочее колесо, не может быть использована турбиной и представляет собой потерю. Поэтому величину этой энергии всегда стремятся уменьшить и за счёт этого увеличить коэффициент полезного действия   турбины. Это достигается в отсасывающей трубе, представляющей собой плавно расширяющийся диффузор.

Величину удельной энергии $e_2$ можно изменять за счёт её составляющих (см. ф. \ref{eq_fx}). Величину скорости $V_2$ не уменьшить, поскольку она зависит от турбины, следовательно, можно уменьшить величину $z_2$, уменьшая высоту установки  турбины над уровнем нижнего бьефа, а также можно понизить давление $p_2$.

Наличие отсасывающей трубы и позволяет добиться понижения давления $p_2$ в сечении 2-2. Для того чтобы проиллюстрировать это, составим уравнение Бернулли для сечений 2-2 и 5-5 (выходного сечения отсасывающей трубы):

\begin{equation}
\label{eq_fy}
   \frac{p_2}{\rho g} + z_2 + \frac{\alpha_2 V_2^2 }{2 g} = \frac{p_5}{\rho g} - z_5 + \frac{\alpha_5 V_5^2 }{2 g} + h_{f_{2-5}} \, \, ,
\end{equation}

где $h_{f_{2-5}}$ -- гидравлические потери в отсасывающей трубе.

\vspace{0.5 cm}


Будем считать, что в сечении 5-5 давление распределено статически (то есть подчиняется основному уравнению гидростатики). Это допущение можно обосновать тем, что на выходе из отсасывающей трубы скорости потока относительно невелики и равномерно распределены по всей площади сечения. Тогда на глубине $h_5 = z_5$ величина избыточного давления равна:

$$
   p_5 = \rho g h_5 =  \rho g z_5 \, \, ,
$$

откуда

\begin{equation}
\label{eq_fz}
   \frac{p_5}{\rho g} = z_5
\end{equation}


Тогда, с учётом принятого допущения (\ref{eq_fz}), выражение (\ref{eq_fy}) примет вид:

$$
   \frac{p_2}{\rho g} + z_2 + \frac{\alpha_2 V_2^2 }{2 g} = z_5 - z_5 + \frac{\alpha_5 V_5^2 }{2 g} + h_{f_{2-5}}
$$

$$
   \frac{p_2}{\rho g} + z_2 + \frac{\alpha_2 V_2^2 }{2 g} = \frac{\alpha_5 V_5^2 }{2 g} + h_{f_{2-5}}
$$

Выразим из последнего полученного выражения пьезометрический напор $\frac{p_2}{\rho g}$ за рабочим колесом:

\begin{equation}
\label{eq_fzb}
   \frac{p_2}{\rho g} = \frac{\alpha_5 V_5^2 }{2 g} + h_{f_{2-5}} - z_2 - \frac{\alpha_2 V_2^2 }{2 g}
\end{equation}

или

$$
   \frac{p_2}{\rho g} = - \left( z_2 + \frac{\alpha_2 V_2^2 }{2 g} - \frac{\alpha_5 V_5^2 }{2 g} - h_{f_{2-5}} \right)
$$

Это выражение показывает, что {\color{red}\textit{отсасывающая труба уменьшает давление за рабочим колесом и создаёт здесь вакуум}}, величину которого можно разложить на две части:

$ z_2$ (можно считать равным высоте отсасывания $H_s$) -- статическое изменение давления (статический вакуум\footnote{Как будет показано далее статическая составляющая пьезометрического напора может быть как положительной -- тогда она стремится понизить давление в сечении 2-2, образуя вакуум, так и отрицательной -- тогда эта составляющая повышает давление в сечении 2-2.}), обусловленное разностью отметок сечений 2-2 и 5-5.

\vspace{0.5 cm}

$ \Delta h_{\text{дин.}} = \cfrac{\alpha_2 V_2^2 }{2 g} - \cfrac{\alpha_5 V_5^2 }{2 g} - h_{f_{2-5}} $ -- динамическое понижение давления, которое создаётся за счёт преобразования или, как говорят ``восстановления'' кинетической энергии $\frac{\alpha_2 V_2^2 }{2 g}$ в потенциальную диффузорным (обусловленным увеличением площади поперечного сечения вдоль пути следования потока) воздействием отсасывающей трубы.

Основной смысл воздействия отсасывающей трубы на поток состоит именно в динамическом понижении давления за рабочим колесом.

\vspace{1 cm}

Рассмотрим как изменяется величина удельной энергии $e_2$ на выходе из рабочего колеса турбины под воздействием отсасывающей трубы. Для этого в выражение (\ref{eq_fx}) подставим выражение (\ref{eq_fzb}):

$$
   e_2 = \left[ \frac{\alpha_5 V_5^2 }{2 g} + h_{f_{2-5}} - z_2 - \frac{\alpha_2 V_2^2 }{2 g} \right] + z_2 + \frac{\alpha_2 V_2^2 }{2 g}
$$

$$
   e_2 = \frac{\alpha_5 V_5^2 }{2 g} + h_{f_{2-5}} 
$$

Получился очень важный результат. Оказывается, что при наличии отсасывающей трубы теряемая турбиной энергия $e_2$ состоит из выходных потерь $\frac{\alpha_5 V_5^2 }{2 g}$ и гидравлических потерь в отсасывающей трубе.

Для уменьшения теряемой энергии можно увеличить площадь выходного сечения отсасывающей трубы, что приводит к снижению $V_5$, однако при этом необходимо, чтобы потери $h_{f_{2-5}}$ не слишком возрастали. На основании этого выбираются очертания отсасывающей трубы. 



\subsubsection{Допустимая высота отсасывания.}

Как было показано ранее, в сечении выхода из рабочего колеса турбины с отсасывающей трубой возникает область пониженного давления (вакуум). При этом в некоторых частях проточного тракта турбины давление может опускаться ещё ниже.

При обтекании потоком жидкости лопастей рабочего колеса на их тыльной стороне возникает область пониженного давления, причём величина понижения давления (глубина вакуума) тем больше, чем выше частота вращения рабочего колеса. Это следует из уравнения Бернулли: если рассматривать абсолютную скорость движения частиц жидкости относительно поверхности рабочего колеса, то с увеличением этой скорости давление в потоке должно уменьшаться.

Частота вращения рабочего колеса определяется режимом работы турбины. Режим работы турбины устанавливается изменением открытия лопаток направляющего аппарата, а также для поворотно-лопастных и диагональных турбин -- изменением угла установки лопастей рабочего колеса. Степень открытия направляющего аппарата и угол установки лопастей выбираются в зависимости от напора, поэтому можно сказать, что частота вращения рабочего колеса есть функция напора.

Таким образом, в некоторой части проточного тракта гидравлической турбины давления могут оказываться низкими настолько, что это может привести к возникновению кавитации.

\begin{opred}
   Кавитация (от лат. kavitas -- пустота) -- процесс парообразования и последующей конденсации пузырьков в потоке жидкости, сопровождающийся шумами и гидравлическими ударами от схлопывания пузырьков.
\end{opred}

 Кавитация (начало роста пузырьков) наблюдается на участках потока, в которых величина давления оказывается меньше давления насыщенных паров $p_{\text{н.п.}}$

При работе турбины стараются избежать возникновения кавитации, так как в результате многочисленных ударов (перепадов давления) от схлопывания пузырьков возникают разрушения обтекаемых поверхностей, а сами пузырьки (кавитационные каверны) создают стеснения живых сечений, в конечном счёте уменьшая к.п.д. турбины. Зная условие возникновения кавитации можно сформулировать обратное условие. Кавитация не будет возникать в потоке, давление в каждой точке которого выше давления насыщенного пара этой жидкости. Формально условие отсутствия кавитации можно записать:

\begin{equation}
\label{eq_fzc}
   p_i > p_{\text{н.п.}}
\end{equation}
где $p_i$ -- абсолютное давление в некотором, рассматриваемом сечении потока (проточного тракта турбины).

$p_{\text{н.п.}}$ -- давление насыщенных паров жидкости (в общем случае зависит от температуры и от рода жидкости).

\vspace{0.5 cm}

То есть для бескавитационной работы турбины необходимо поддерживать в её проточном тракте определённую величину давления $p_i$ и предусматривать такие условия, при которых это давление не опускалось бы ниже некоторой критической величины.

\vspace{0.5 cm}

При рассмотрении условий работы отсасывающей трубы было показано, что её наличие создаёт за рабочим колесом динамическое понижение давления. Повысить давление в этом сечении можно за счёт статического давления, увеличив глубину погружения турбины с отсасывающей трубой под поверхность воды в нижнем бьефе, то есть уменьшить высоту отсасывания $H_s$.

Таким образом, бескавитационная работа турбины с выбранной отсасывающей трубой может быть обеспечена подбором высоты отсасывания. Такой подбор можно выполнить с использованием зависимости (\ref{eq_fzb}), задаваясь необходимой величиной давления в сечении 2-2 (см. рис.\ref{ppc}) за рабочим колесом.

Перепишем выражение (\ref{eq_fzb}). Без серьёзной погрешности можно принимать $z_2 = H_s$ ($H_s$ -- высота отсасывания).


$$
   \frac{p_2}{\rho g} = \frac{\alpha_5 V_5^2 }{2 g} + h_{f_{2-5}} - H_s - \frac{\alpha_2 V_2^2 }{2 g}
$$

Это выражение позволяет определить величину избыточного давления за рабочим колесом. Поскольку гидроэлектростанции могут возводиться на различных высотах относительно уровня моря, величина атмосферного давления для разных ГЭС оказывается различной. Чтобы учесть это обстоятельство при выборе высоты отсасывания $H_s$ оперируют не избыточным, а абсолютным давлением. Для определения абсолютного давления в сечении 2-2 необходимо к величине избыточного давления прибавить фактическое атмосферное давление на соответствующей высотной отметке уровня воды в нижнем бьефе $p_{\text{атм.}}^{\text{н.б.}}$ , итак:

$$
   p_{\text{абс}_2} = p_{\text{атм.}}^{\text{н.б.}} + p_2
$$

$$
   \frac{p_{\text{абс}_2}}{\rho g} = \frac{p_{\text{атм.}}^{\text{н.б.}}}{\rho g} + \frac{\alpha_5 V_5^2 }{2 g} + h_{f_{2-5}} - H_s - \frac{\alpha_2 V_2^2 }{2 g}
$$

\vspace{0.5 cm}

так как $  \cfrac{\alpha_2 V_2^2 }{2 g} - \cfrac{\alpha_5 V_5^2 }{2 g} - h_{f_{2-5}} = \Delta h_{\text{дин.}} $ -- динамическое понижение давления:

$$
   \frac{p_{\text{абс}_2}}{\rho g} = \frac{p_{\text{атм.}}^{\text{н.б.}}}{\rho g} - H_s - \Delta h_{\text{дин.}}
$$

Исходя из законов подобия режимов работы турбин одного типа следует, что величина динамического понижения давления пропорциональна напору:

$$ 
   \Delta h_{\text{дин.}} = \sigma_{\text{отс.}} \cdot H \, ,
$$
где $\sigma_{\text{отс.}}$ -- коэффициент кавитации, который характеризует относительное динамическое понижение давления в сечении 2-2, создаваемое отсасывающей трубой.

\vspace{0.5 cm}

Тогда выражения для величины абсолютного давления в сечении 2-2:

$$
   \frac{p_{\text{абс}_2}}{\rho g} = \frac{p_{\text{атм.}}^{\text{н.б.}}}{\rho g} - H_s - \sigma_{\text{отс.}} H 
$$

Условие отсутствия кавитации в сечении 2-2 в соответствии с (\ref{eq_fzc}):

$$
   \frac{p_{\text{абс}_2}}{\rho g} > \frac{p_{\text{н.п.}}}{\rho g} \, ,
$$

$$
   \frac{p_{\text{атм.}}^{\text{н.б.}}}{\rho g} - H_s - \sigma_{\text{отс.}} H > \frac{p_{\text{н.п.}}}{\rho g}
$$

Откуда:

\begin{equation}
\label{eq_fzd}
   H_s \leqslant \frac{p_{\text{атм.}}^{\text{н.б.}}}{\rho g} - \frac{p_{\text{н.п.}}}{\rho g} - \sigma_{\text{отс.}} H
\end{equation}

Значит для отсутствия кавитации в сечении 2-2 необходимо ограничивать высоту отсасывания $H_s$. То есть нельзя располагать турбину слишком высоко по отношению к нижнему бьефу.

\vspace{0.5 cm}

Поскольку проточном тракте рабочего колеса величина давления может опускаться ниже величины давления в сечении 2-2, необходимо обеспечивать бескавитационную работу турбины именно по минимальному давлению в проточном тракте.

Динамическое понижение давления в различных точках рабочего колеса также пропорционально напору и может быть описано с помощью различных (для разных точек на поверхности лопастей рабочего колеса) коэффициентов кавитации $\sigma$. То есть вышеприведённые рассуждения справедливы также и ко всем точкам рабочего колеса.

Таким образом, для отсутствия кавитации в турбине также необходимо ограничивать высоту отсасывания, причём для описания динамического понижения давления выбирать коэффициент кавитации $\sigma$, характеризующий точки на рабочем колесе с минимальным давлением. Следовательно высота отсасывания должна удовлетворять неравенству:

\begin{equation}
\label{eq_fze}
   H_s \leqslant \frac{p_{\text{атм.}}^{\text{н.б.}}}{\rho g} - \frac{p_{\text{н.п.}}}{\rho g} - \sigma H
\end{equation}

На практике величины, входящие в правую часть этого неравенства определяются по эмпирическим зависимостям.

Атмосферное давление зависит от отметки над уровнем моря $\nabla$, и приближённо эта зависимость описывается формулой:

$$
   \frac{p_{\text{атм.}}}{\rho g} = 10.33 - \frac{\nabla}{900} \, \, ,
$$
где $\cfrac{p_{\text{атм.}}}{\rho g}$ -- пьезометрический напор, соответствующий атмосферному давлению на отметке $\nabla$, [м];

$\nabla$ -- высота над уровнем моря [м], в Балтийской системе высот.

\vspace{0.5 cm}

Давление насыщенных паров воды зависит от температуры,в приближённых расчётах с достаточной точностью и запасом для холодной воды можно принимать:

$$
   \frac{p_{\text{н.п.}}}{\rho g} = 0.33 [\text{м}]
$$
\vspace{0.5 cm}

Для более точных расчётов давление насыщенных паров принимается в соответствии с расчётной температурой потока воды, поступающего через проточный тракт турбины по таблице \ref{tab_2}. 



\begin{table}[ht]
\caption{Зависимость давления насыщенных паров воды от температуры.}
\label{tab_2}
\centering

\begin{tabular}{|c|c|}
\hline
  Температура паров (воды), $^\circ$C & $\cfrac{p_{\text{н.п.}}}{\rho g}$, м вод ст.\\
   & \\ 
\hline
  0 & 0.06 \\
  5 & 0.09 \\
  10 & 0.12 \\
  20 & 0.24 \\
  30 & 0.43 \\
  40 & 0.72 \\
  50 & 1.26 \\
  60 & 2.03 \\
  70 & 3.18 \\
  80 & 4.83 \\
  90 & 7.15 \\
  100 & 10.33 \\
\hline
\end{tabular}
\end{table}

При выборе давления насыщенных паров следует руководствоваться наибольшей из возможных расчётных температур потока воды, так как с повышением температуры давление насыщенных паров возрастает и кавитация возникает в потоке, при более высоких значениях абсолютного давления (то есть при меньшей величине вакуума в проточной части рабочего колеса).

\vspace{0.5 cm}

Коэффициент кавитации определяется опытным путём на специальной модельной установке. Модельная установка работает при постоянном напоре в определённом режиме. Она представляет собой модель турбины выбранного типа. Конструкция модельной установки позволяет изменять величину давления над нижним бьефом $p_{\text{атм.}}^{\text{н.б.}}$ Коэффициент кавитации находится из выражения (\ref{eq_fze}) на основании серии испытаний, в ходе которых постепенно понижается давление в нижнем бьефе. При определённых значениях давления $p_{\text{атм.}}^{\text{н.б.}}$ возникает кавитация, что сопровождается появлением шума, вибрации, а также падения мощности на валу модели турбины -- кавитационный срыв характеристик. Величина коэффициента кавитации $\sigma$ непосредственно при срыве представляет собой критический коэффициент кавитации $\sigma_{\text{кр.}}$

При расчётах допустимой высоты отсасывания $H_s$ расчётный коэффициент кавитации  турбины определяют по критическому значению $\sigma_{\text{кр.}}$ с введением коэффициента запаса $k_{\sigma}$, который принимается ($k_{\sigma} = $ 1.1 --- 1.2).

$$
   \sigma = k_{\sigma} \cdot \sigma_{\text{кр.}}
$$

\vspace{0.5 cm}

Таким образом, неравенство (\ref{eq_fze}) для определения высоты отсасывания принимает вид:

$$
   H_s \leqslant 10.33 - \frac{\nabla}{900} - 0.33 - \sigma H
$$

$$
   H_s \leqslant 10 - \frac{\nabla}{900} - \sigma H
$$

Это неравенство позволяет определить допустимую величину отсасывания.

\begin{opred}
Допустимая высота отсасывания -- минимальная высота размещения рабочего колеса турбины в здании ГЭС по отношению к нижнему бьефу, при которой работа турбины проходит в бескавитационном режиме.
\end{opred}

И выражение для определения допустимой высоты отсасывания:

\begin{equation}
\label{eq_fzf}
   H_s = 10 - \frac{\nabla}{900} - \sigma H
\end{equation}


\vspace{1 cm}


\begin{primer}
\label{Primer1}
 Определить допустимую высоту отсасывания для турбины с напором $H = 60$ м, при отметке нижнего бьефа $360$ м, если величина критического коэффициента кавитации $\sigma_{\text{кр.}} = 0.1$.


\textbf{Решение:} Задаёмся величиной коэффициента запаса $k_{\sigma} = 1.15$. Тогда расчётное значение коэффициента кавитации:

$$
   \sigma = 1.15 \cdot 0.1 = 0.115
$$

По выражению (\ref{eq_fzf}) вычисляем $H_s$.

$$
   H_s = 10 - \frac{360}{900} - 0.115 \cdot 60 = 2.7 \text{м}
$$

Турбину можно установить на $2.7$ м выше отметки нижнего бьефа.
\end{primer}

\vspace{0.5 cm}



\begin{primer}
\label{Primer2}
   Какова будет допустимая высота отсасывания, если для условий из предыдущего примера применить другую турбину, у которой $\sigma_{\text{кр.}} = 0.2$?

\textbf{Решение:} При том же значении коэффициента запаса:

$$
   \sigma = 1.15 \cdot 0.2 = 0.23
$$

И высота отсасывания по (\ref{eq_fzf}):

$$
   H_s = 10 - \frac{360}{900} - 0.23 \cdot 60 = -4.2 \text{м}
$$

Такую турбину придётся установить так, чтобы она была заглублена под уровень нижнего бьефа на $4.2$ м.
\end{primer}

\vspace{1 cm}


Таким образом, при положительных значениях $H_s$ турбину можно располагать выше уровня нижнего бьефа. При отрицательных значениях $H_s$ турбину заглубляют под уровень нижнего бьефа, что приводит к увеличению объёмов грунтовых и бетонных работ и повышению затрат на строительство ГЭС.

\begin{center}
  Способы отсчёта высоты отсасывания.
\end{center}

Для различных типов турбин высота отсасывания определяется по разному:

\begin{description}
\item[1.] Для вертикальных радиально-осевых и диагональных турбин $H_s$ отсчитывается от нижней кромки направляющего аппарата.


\item[2.] Для вертикальных осевых турбин $H_s$ отсчитывается от оси поворота лопастей рабочего колеса.



\item[3.] У горизонтальных осевых турбин $H_s$ отсчитывается от верхней точки рабочего колеса.
\end{description}



\begin{figure} [ht]
  \center
  \includegraphics [scale = 0.8] {ppd}  \caption{Отсчёт высоты отсасывания для вертикальных радиально-осевых и диагональных турбин.}
  \label{ppd}
\end{figure}



\begin{figure} [ht]
  \center
  \includegraphics [scale = 0.8] {ppe}  \caption{Отсчёт высоты отсасывания для вертикальных поворотно-лопастных и пропеллерных турбин.}
  \label{ppe}
\end{figure}



\begin{figure} [ht]
  \center
  \includegraphics [scale = 0.8] {ppf}  \caption{Отсчёт высоты отсасывания для горизонтальных турбин.}
  \label{ppf}
\end{figure}




%%_________________________________________________________________________________________________________________________
%%_________________________________________________________________________________________________________________________
%%_________________________________________________________________________________________________________________________



\section[Элементы конструкции ГЭС.]{Элементы конструкции ГЭС и расчёт их основных параметров.}

В основе определения размеров элементов конструкции ГЭС лежит геометрическое подобие, пропорция. ГЭС могут быть относительно большими, или относительно малыми, но размеры всех частей их элементов соотносятся друг с другом определённым образом. К примеру у одной из типовых отсасывающих труб длина в два раза превышает высоту. Тогда при проектировании, зная заранее высоту этой отсасывающей трубы, по известной для этой трубы пропорции определяют её длину.

Поскольку основным элементом ГЭС является гидравлическая   турбина, отправной  точкой отсчёта размеров (базисным размером) считается  диаметр рабочего колеса $D_1$. Размеры элементов конструкции задаются не абсолютными, а относительными. Относительные размеры элементов ГЭС -- это такие размеры, которые имели бы соответствующие элементы ГЭС с диаметром рабочего колеса турбины $D_1 = 1$ м. Для определения действительных размеров этих элементов необходимо умножить из относительный размер на величину $D_1$ применённой в конструкции ГЭС турбины. 

Относительные размеры обозначают горизонтальной чертой над символом, соответствующим размеру элемента конструкции. 

Из вышесказанного следует, что принципиальным моментом при проектировании ГЭС является выбор типа турбины и установление размеров её рабочего колеса. Проектирование ГЭС можно рассматривать как процесс, состоящий из следующих этапов:

\begin{enumerate}
\item Определение величины напора, расхода и мощности будущей ГЭС по гидрологическим параметрам водотока -- определение гидропотенциала водотока. 
\item Предварительное назначение количества гидроагрегатов, величины средневзвешенного напора и расчётного напора.
\item Выбор и определение основных параметров гидроагрегатов (при выполнении этого этапа становится известной величина диаметра рабочего колеса турбины $D_1$).
\item Установление размеров остальных частей конструкции здания ГЭС в соответствии с выбранной  турбиной.

\subitem 4.1. Выбор (расчёт) спиральной камеры.
\subitem 4.2. Выбор размеров отсасывающей трубы.

\item Уточнение принятых параметров и их взаимосвязей
\item Окончательная компоновка здания ГЭС.
\end{enumerate}



%%_________________________________________________________________________________________________________________________
%%_________________________________________________________________________________________________________________________
%%_________________________________________________________________________________________________________________________

\subsection{Генератор.}

\subsubsection{Преобразование механической энергии в электрическую.}

При движении тока I (заряженных частиц) по проводнику, вокруг него появляется круговое магнитное поле без выраженных полюсов.

\begin{figure} [ht]
  \center
  \includegraphics [scale = 0.45] {ppj}
  \caption{Магнитное поле вокруг проводника с электрическим током.}
  \label{img_ppj}
\end{figure}

Магнитное поле иллюстрируется силовыми магнитными линиями, направленными от северного магнитного полюса к южному. С удалением от магнита (проводника) величина магнитного поля уменьшается.

Направление магнитного поля вокруг проводника с током определяется по правилу правого винта (буравчика): если вращать буравчик в направлении силовых линий магнитного поля (завихрения магнитного поля), то он будет ввинчиваться в направлении движения  тока в проводнике (направления вектора вихря этого поля\footnote{Здесь имеется аналогия между характеристиками магнитного поля и гидродинамическими параметрами поля скоростей потока жидкости.(Сведения только для интереса, использовать здесь эту аналогию в дальнейшем мы не будем.)}).

При сближении двух магнитов одноимёнными полюсами, они отталкиваются, взаимодействуя своими полями.

\begin{figure} [ht]
  \center
  \includegraphics [scale = 0.9] {pph}
  \caption{Взаимодействие двух магнитов.}
  \label{img_pph}
\end{figure}

При сближении двух магнитов разноимёнными полюсами, они притягиваются друг к другу.

При прохождении электрического тока по проводнику, свёрнутому в виде спирали, магнитное поле, складываясь, образует на концах спирали полюса -- северный и южный. Если в середину такой спирали (катушки) поместить сердечник из малоуглеродистой стали, обладающей хорошей магнитной проводимостью, то образуется электромагнит, имеющий все свойства магнита.

\begin{figure} [ht]
  \center
  \includegraphics [scale = 0.9] {ppi}
  \caption{Электромагнит.}
  \label{img_ppi}
\end{figure}

\newpage

Прямоугольную рамку L из проводника, в котором протекает электрический ток I (Рис. \ref{img_ppk} (а)) также можно рассматривать как магнит. Полюса этого магнита расположены на оси, нормальной к плоскости рамки L.

\begin{figure} [ht]
  \center
  \includegraphics [scale = 0.9] {ppk}
  \caption{Принципиальная схема синхронной электрической машины.}
  \label{img_ppk}
\end{figure}

Если поместить такую рамку, через которую протекает электрический ток, во внешнее магнитное поле С-Ю, созданное, например, постоянным магнитом, то эту систему можно рассматривать как два взаимодействующих между собой магнита (Рис. \ref{img_ppk} (б)). Так как одноимённые полюса магнитов отталкиваются, на рамку будет действовать пара сил, поворачивая её.

Вращающаяся рамка называется ротором, неподвижные магниты, создающие магнитное поле -- статором.

Если такую рамку L без электрического тока вращать в магнитном поле, то есть сообщить ей вращательный момент, то в соответствии с законом электромагнитной индукции в проводнике рамки возникнет электрический ток. Величина и направление этого индуцированного тока стремятся создать такое магнитное поле вокруг проводника рамки, которое препятствовало бы перемещению этой рамки (то есть вращение рамки, изображённой на рисунке \ref{img_ppk} против часовой стрелки приведёт к возникновению в ней тока I с таким направлением, как показано на рисунке). Этот ток возникает в результате э.д.с. (электродвижущая сила) индукции. При вращении рамки с постоянной скоростью этот ток будет переменным. 

И, наоборот, если по проводнику рамки L пропускать переменный электрический ток, она будет равномерно вращаться с частотой, равной частоте электрического тока.

Такая электрическая машина называется обратимой, поскольку может работать и в режиме двигателя и в режиме генератора. Также такие машины называются синхронными, потому что частота их вращения совпадает с частотой переменного электрического тока.

Для преобразования энергии в электрических машинах может быть задействовано несколько пар полюсов.


%%_________________________________________________________________________________________________________________________
%%_________________________________________________________________________________________________________________________
%%_________________________________________________________________________________________________________________________

\subsubsection{Гидрогенератор.}

 \begin{opred}
   Гидрогенератор -- электрическая машина для выработки электроэнергии на гидроэлектростанции. 
 \end{opred}

 Обычно гидрогенератор представляет собой синхронную явнополюсную электрическую машину, приводимую во вращение от гидротурбины. Конструкция гидрогенератора в основном определяется параметрами гидравлической турбины. Объединённую конструкцию гидравлической турбины и гидрогенератора называют  гидроагрегат.

Мощность генератора должна соответствовать мощности, развиваемой на валу турбины.

Номинальная мощность генератора может быть определена по выражению:

$$
   P = \eta_{\text{ген.}} \cdot N_\text{в} = N_{\text{г.а.}} \quad , [\text{Вт}],
$$
где $\eta_{\text{ген.}}$ -- к.п.д. генератора;

$N_\text{в}$ -- мощность на валу турбины, [Вт].

\vspace{0.5cm}

Номинальная мощность P это активная (действительная) мощность генератора, однако его размеры определяются величиной полной (кажущейся) мощности S, связанной с активной мощностью соотношением:

$$
   P = S \cdot \cos \varphi = \sqrt{3} \cdot U \cdot I \cdot \cos \varphi \, ,
$$
где $\sqrt{3}$ -- коэффициент, характеризующий трёхфазную систему;

$U$ -- номинальное линейное напряжение на зажимах генератора, [В];

$I$ -- номинальная сила тока, [А];

$\cos \varphi$ -- номинальный коэффициент мощности, характеризующий отставание по фазе периодических изменений тока, от периодических изменений напряжения.

\vspace{0.5cm}

При отставании периодических колебаний тока от колебаний напряжения по фазе на четверть периода $\varphi = 90^{\circ}$, то есть когда мгновенное напряжение достигает наибольшей величины, а значение тока в этот момент равно нулю, генератор не выдаёт активной мощности и не требует от турбины энергии для своего вращения. Величина $\cos \varphi$ при этом равна нулю. 

Обычно генераторы строятся на номинальный коэффициент мощности $\cos \varphi = 0.8$, если в технических условиях не оговорено иное его значение. Величина $\cos \varphi$ зависит от потребителей электрической энергии и устанавливается в связи с условиями работы генератора в энергосистеме на основании специальных электрических расчётов.

Для генераторов установлены стандартные величины вырабатываемого ими напряжения.

Для поддержания в электрической сети принятой в России стандартной частоты переменного тока 50 Герц вал генератора должен вращаться с определённой частотой, зависящей от числа полюсов генератора. Поскольку гидравлическая турбина и генератор находятся на одном валу это условие налагает требования для частоты вращения рабочего колеса турбины.

Частота переменного тока f связана с числом пар полюсов генератора и частотой вращения его ротора n следующим соотношением:

$$
  f = \frac{p_p \cdot n}{60} \, ,
$$
 где $f$ -- частота переменного тока, [Гц];

$p_p$ -- число пар полюсов генератора;

$n$ -- частота вращения вала генератора, $\left[ \frac{\text{об}}{\text{мин}} \right]$.

\vspace{0.5cm}

Таким образом, при различном количестве пар полюсов генератора для обеспечения необходимой частоты переменного тока существует ряд рекомендуемых значений частоты вращения вала гидроагрегата:

\vspace{0.5cm}

\label{riad}

46.9; 50; 51.7; 53.5; 55.6; 57.7; 60; 62.5; 68.2; 71.4; 75; 83.3; 85.7; 88.2; 93.7; 100; 107.1; 115.4; 125; 136.4; 142.8; 150; 166.7; 187.5; 200; 214.3; 230.8; 250; 300; 333.3; 375; 428.6; 500; 600; 750.



%%_________________________________________________________________________________________________________________________
%%_________________________________________________________________________________________________________________________
%%_________________________________________________________________________________________________________________________

\subsection{Гидравлическая турбина.}


\subsubsection[Законы подобия режимов турбин.]{Турбины одного типа и законы подобия режимов их работы.}

Гидравлические турбины могут выполняться самых различных размеров и работать в широком диапазоне напоров и расходов. Поэтому необходимо уметь сравнивать их между собой для выбора оптимального варианта.

Тип турбины определяется формой (геометрией) её проточного тракта. Иными словами все турбины данного типа имеют геометрически подобные формы проточной части. Следует подчеркнуть, что размер турбины не участвует в определении её типа. 

Понятие типа турбины обусловлено условиями её проектирования. Вначале проектирования выполняется расчёт формы турбины определённого размера, например с диаметром рабочего колеса $D_1 = 1$ м. Затем изготавливается модель этой турбины и на испытательном стенде снимаются её характеристики. Но для реальных ГЭС эта турбина может оказаться слишком маленькой, или, наоборот, слишком большой. Тогда размеры этой турбины изменяют. Необходимо знать какие характеристики будет иметь эта турбина с другими размерами. Такие знания обеспечивает теория подобия гидравлических турбин.

\vspace{0.5cm}

Две турбины одного типа, но различного размера являются геометрически подобными, если удовлетворяются следующие условия:

\begin{description}
\item[-] Все соответствующие углы у них равны.
\item[-] Отношение всех соответствующих размеров постоянно:

\begin{equation}
\label{eq_fzj}
   \frac{D_{1_1}}{D_{1_2}} = \frac{D_{2_1}}{D_{2_2}} = \frac{b_{0_1}}{b_{0_2}} = \cdots
\end{equation}

\end{description}

Вторые индексы обозначают принадлежность размера к одной из сравниваемых турбин: либо к турбине 1, либо к турбине 2.

\vspace{0.5cm}

Режим работы турбины определяется двумя параметрами: расходом Q и частотой вращения рабочего колеса n.

Режимы турбин одного типа будут подобны, если сохраняется геометрическое подобие параллелограммов, или треугольников скоростей в соответствующих точках проточного тракта.

Следовательно при подобии режимов:

\begin{description}
\item[-] Направления всех соответствующих скоростей должны быть одинаковыми, то есть соответственные углы, характеризующие направление средней скорости равны:

\begin{equation}
\label{eq_fzh}
  \alpha_{1 1} = \alpha_{1 2} \, , \quad  \alpha_{2 1} = \alpha_{2 2} \, , \quad \dots 
\end{equation}

\item[-] Отношения всех соответствующих скоростей должны быть постоянными:

\begin{equation}
\label{eq_fzi}
   \frac{V_{1_1}}{V_{1_2}} = \frac{U_{1_1}}{U_{1_2}} = \frac{\omega_{1_1}}{\omega_{1_2}} = \frac{V_{2_1}}{V_{2_2}} = \frac{U_{2_1}}{U_{2_2}} = \cdots
\end{equation}

\end{description}

На основании этих условий подобия (геометрического подобия и подобия режимов) установим закономерности, определяющие соотношение между параметрами турбин одного типа в подобных режимах их работы.

\vspace{1 cm}

Пусть имеются две турбины одного типа, но различного размера $D_1$ и $D_2$. Углы открытия направляющего аппарата и установки лопастей рабочего колеса равны: $\alpha_{0_1} = \alpha_{0_2}$ и $\varphi_{0_1} = \varphi_{0_2}$ .

Каковы должны быть соотношения частот вращения $n_1$ и $n_2$, расходов $Q_1$ и $Q_2$ и мощностей $N_1$ и $N_2$, чтобы режимы работы обеих турбин были подобными?

\vspace{0.5cm}

Для решения этой задачи прежде всего используем геометрические (\ref{eq_fzj}) и кинематические (\ref{eq_fzh}), (\ref{eq_fzi}) соотношения, характеризующие подобие режимов работы рассматриваемых турбин.

Рассмотрим как соотносятся между собой линейные скорости движения точки, находящейся на некотором расстоянии от оси вращения на рабочем колесе (эта точка при вращении описывает окружность диаметром $D_i$). Линейная скорость этой точки:

$$
   U_i = \pi \cdot D_i \cdot n
$$

Таким образом, отношение линейных скоростей таких точек на рабочем колесе каждой из рассматриваемых турбин:

\begin{equation}
\label{eq_fzk}
   \frac{U_{i_1}}{U_{i_2}} = \frac{\pi \cdot D_{i_1} \cdot n_1}{\pi \cdot D_{i_2} \cdot n_2} = \frac{D_{i_1} \cdot n_1}{D_{i_2} \cdot n_2}
\end{equation}

Теперь рассмотрим как соотносятся радиальные составляющие средних скоростей потока. В соответствии с выражением (\ref{eq_ff}) величина радиальной составляющей скорости зависит от расхода и площади цилиндрического сечения направляющего аппарата\footnote{В качестве этой скорости можно рассматривать любую скорость потока воды в проточном тракте рассматриваемых турбин, величина которой зависит от расхода и площади поперечного сечения, поскольку условия кинематического подобия должны выполняться для скоростей в любой точке проточного тракта рабочего колеса. Радиальная составляющая скорости на выходе из направляющего аппарата взята здесь из соображений удобства.}.

$$
   V_{0r} = \frac{Q_{\text{а}}}{\pi \cdot D_{02} \cdot b_0}
$$
где $Q_{\text{а}}$ -- расход, проходящий через гидроагрегат;

$D_{02}$ -- диаметр окружности по выходным кромкам направляющего аппарата;

$b_0$ -- высота направляющего аппарата.

\vspace{0.5cm}

Тогда соотношение радиальных составляющих средних скоростей на выходе из направляющего аппарата:

\begin{equation}
\label{eq_fzl}
   \frac{V_{0r_1}}{V_{0r_2}} = \frac{\cfrac{Q_{\text{а}_1}}{\pi D_{02_1}  b_{0_1} }}{\cfrac{Q_{\text{а}_2}}{\pi D_{02_2}  b_{0_2} }} = \frac{Q_{\text{а}_1} D_{02_2}  b_{0_1}}{Q_{\text{а}_2} D_{02_1}  b_{0_2}}
\end{equation}

Поскольку геометрическое подобие выполняется, то согласно условию (\ref{eq_fzj}):

\begin{equation}
\label{eq_fzn}
   \frac{D_{02_2}}{D_{02_1}} = \frac{b_{0_2}}{b_{0_1}} = \frac{D_{i_2}}{D_{i_1}} = \frac{D_2}{D_1} \, ,
\end{equation}
где $D_1$ -- диаметр, соответствующий положению некоторой точки на рабочем колесе турбины 1.

$D_2$ -- диаметр, соответствующий диаметру $D_1$ на рабочем колесе турбины 2.

\vspace{0.5cm}

В качестве характерных диаметров $D_1$ и $D_2$ могут быть приняты любые два геометрически подобных диаметра, например диаметр рабочего колеса $D_1$, тогда $D_1 = D_{1_1}$, $D_2 = D_{1_2}$.

И выражение (\ref{eq_fzl}) можно переписать в виде:

\begin{equation}
\label{eq_fzm}
   \frac{V_{0r_1}}{V_{0r_2}} = \frac{Q_{\text{а}_1} \cdot {D_2}^2  }{Q_{\text{а}_2} \cdot {D_1}^2 }
\end{equation}

Поскольку при сохранении режимов (подобии режимов) должно выполняться равенство отношений скоростей (условие (\ref{eq_fzi})), то можно написать:

$$
    \frac{U_{i_1}}{U_{i_2}} = \frac{V_{0r_1}}{V_{0r_2}}
$$

Или, в соответствии с (\ref{eq_fzk}) и (\ref{eq_fzm}): 

$$
   \frac{D_{i_1} \cdot n_1}{D_{i_2} \cdot n_2} = \frac{Q_{\text{а}_1} \cdot {D_2}^2  }{Q_{\text{а}_2} \cdot {D_1}^2 }
$$

Откуда соотношение частот вращения двух турбин (по условиям кинематического подобия):

$$
   \frac{n_1}{n_2} = \frac{Q_{\text{а}_1} \cdot {D_2}^2 \cdot D_{i_2} }{Q_{\text{а}_2} \cdot {D_1}^2 \cdot D_{i_1} }
$$

Или, учитывая (\ref{eq_fzn}):

\begin{equation}
\label{eq_fzo}
   \frac{n_1}{n_2} = \frac{Q_{\text{а}_1} \cdot {D_2}^3  }{Q_{\text{а}_2} \cdot {D_1}^3 }
\end{equation}

Можно обобщить этот результат, преобразовав выражение (\ref{eq_fzo}) так, чтобы с каждой стороны от знака равенства стояли величины, относящиеся к одной из сравниваемых турбин:

$$
   \frac{Q_{\text{а}_1}}{n_1 \cdot {D_1}^3} = \frac{Q_{\text{а}_2}}{n_2 \cdot {D_2}^3} = const \, ,
$$

или

\begin{equation}
\label{eq_fzp}
   \frac{Q}{n \cdot D^3} = const
\end{equation}

Выражение (\ref{eq_fzp}) называется кинематическим условием подобия (сохранения) режима работы гидротурбины.

\vspace{1 cm}

Теперь используем энергетические соотношения, учитывая что сравниваемые турбины работают при разных напорах $H_1$ и $H_2$ и имеют гидравлические коэффициенты полезного действия $\eta_{\text{турб}_1}$ и $\eta_{\text{турб}_2}$ .

Напишем уравнение Эйлера (основное уравнение турбин (\ref{eq_fs})) для каждой из турбин

Для первой турбины:

\begin{equation}
\label{eq_fzq}
  H_1 \cdot \eta_{\text{турб}_1} \cdot g = U_{1_1} \cdot V_{1_1} \cdot \cos{\alpha_{1_1}} - U_{2_1} \cdot V_{2_1} \cdot \cos{\alpha_{2_1}}
\end{equation}

Для второй турбины:

\begin{equation}
\label{eq_fzr}
  H_2 \cdot \eta_{\text{турб}_2} \cdot g = U_{1_2} \cdot V_{1_2} \cdot \cos{\alpha_{1_2}} - U_{2_2} \cdot V_{2_2} \cdot \cos{\alpha_{2_2}}
\end{equation}

Из условий подобия, а также из полученных ранее соотношений для скоростей, кинематические параметры второй турбины, стоящие справа от знака равенства в выражении (\ref{eq_fzr}), можно выразить через кинематические параметры первой турбины. Покажем это.

В соответствии с (\ref{eq_fzk}) и условиями (\ref{eq_fzi}) и (\ref{eq_fzn}), можно написать:

$$
  \frac{U_{i_1}}{U_{i_2}} = \frac{D_1\cdot n_1}{D_2 \cdot n_2} = \frac{U_{1_1}}{U_{1_2}} = \frac{V_{1_1}}{V_{1_2}} = \frac{U_{2_1}}{U_{2_2}} = \frac{V_{2_1}}{V_{2_2}}
$$

Тогда:

$$
   U_{1_2} = \frac{D_2 \cdot n_2}{D_1 \cdot n_1} \cdot U_{1_1} 
$$

$$
   V_{1_2} = \frac{D_2 \cdot n_2}{D_1 \cdot n_1} \cdot V_{1_1}
$$

$$
   U_{2_2} = \frac{D_2 \cdot n_2}{D_1 \cdot n_1} \cdot U_{2_1}
$$

$$
   V_{2_2} = \frac{D_2 \cdot n_2}{D_1 \cdot n_1} \cdot V_{2_1}
$$


После подстановки этих выражений в уравнение Эйлера для второй турбины (\ref{eq_fzr}) оно примет вид:

$$
  H_2 \cdot \eta_{\text{турб}_2} \cdot g = \left( \frac{D_2 \cdot n_2}{D_1 \cdot n_1} \right)^2    \cdot    \left(    U_{1_1} \cdot V_{1_1} \cdot \cos{\alpha_{1_2}} - U_{2_1} \cdot V_{2_1} \cdot \cos{\alpha_{2_2}} \right) 
$$

Из условий подобия (\ref{eq_fzh}) следует что $\alpha_{1_1} = \alpha_{1_2}$ и $\alpha_{2_1} = \alpha_{2_2}$, и, соответственно, $\cos{\alpha_{1_2}} = \cos{\alpha_{1_1}}$, $\cos{\alpha_{2_2}} = \cos{\alpha_{2_1}}$. 
Перепишем ещё раз уравнение Эйлера для второй турбины с учётом этого. 

\begin{equation}
\label{eq_fzs}
   H_2 \cdot \eta_{\text{турб}_2} \cdot g = \left( \frac{D_2 \cdot n_2}{D_1 \cdot n_1} \right)^2    \cdot    \left(    U_{1_1} \cdot V_{1_1} \cdot \cos{\alpha_{1_1}} - U_{2_1} \cdot V_{2_1} \cdot \cos{\alpha_{2_1}} \right) 
\end{equation}

\vspace{0.5cm}

Разделим выражение (\ref{eq_fzq}) на (\ref{eq_fzs}):

$$
  \frac{H_1 \cdot \eta_{\text{турб}_1} \cdot g}{H_2 \cdot \eta_{\text{турб}_2} \cdot g} = \frac{U_{1_1} \cdot V_{1_1} \cdot \cos{\alpha_{1_1}} - U_{2_1} \cdot V_{2_1} \cdot \cos{\alpha_{2_1}}} {\left( \cfrac{D_2 \cdot n_2}{D_1 \cdot n_1} \right)^2    \cdot    \left(    U_{1_1} \cdot V_{1_1} \cdot \cos{\alpha_{1_1}} - U_{2_1} \cdot V_{2_1} \cdot \cos{\alpha_{2_1}} \right)} \, ,
$$

$$
   \frac{H_1 \cdot \eta_{\text{турб}_1} \cdot g}{H_2 \cdot \eta_{\text{турб}_2} \cdot g} = \left( \cfrac{D_1 \cdot n_1}{D_2 \cdot n_2} \right)^2 
$$

Из полученного выражения найдём соотношения между частотами вращения сравниваемых турбин:

\begin{equation}
\label{eq_fzt}
   \boxed{ \frac{n_1}{n_2} = \frac{D_2}{D_1} \cdot \sqrt{\frac{\vphantom{\eta_{\text{турб}_2}} H_1}{ \vphantom{\eta_{\text{турб}_2}} H_2}} \cdot \sqrt{\frac{ \vphantom{1} \eta_{\text{турб}_1}}{\vphantom{1}\eta_{\text{турб}_2}}} }
\end{equation}

\vspace{0.5cm}


Подставив найденное соотношение $\frac{n_1}{n_2}$ в кинематическое условие подобия (\ref{eq_fzp}), получим соотношение расходов:

$$
  \frac{Q_1}{n_1 \cdot {D_1}^3} = \frac{Q_2}{n_2 \cdot {D_2}^3}
$$

$$
  \frac{Q_1}{Q_2} = \frac{n_1 \cdot {D_1}^3}{n_2 \cdot {D_2}^3} = \frac{D_2}{D_1} \cdot \sqrt{\frac{\vphantom{\eta_{\text{турб}_2}} H_1}{ \vphantom{\eta_{\text{турб}_2}} H_2}} \cdot \sqrt{\frac{ \vphantom{1} \eta_{\text{турб}_1}}{\vphantom{1}\eta_{\text{турб}_2}}}  \cdot \frac{{D_1}^3}{{D_2}^3}
$$


\begin{equation}
\label{eq_fzv}
   \boxed{   \frac{Q_1}{Q_2} = \left( \frac{D_1}{D_2} \right)^2 \cdot \sqrt{\frac{\vphantom{\eta_{\text{турб}_2}} H_1}{ \vphantom{\eta_{\text{турб}_2}} H_2}} \cdot \sqrt{\frac{ \vphantom{1} \eta_{\text{турб}_1}}{\vphantom{1}\eta_{\text{турб}_2}}}  }
\end{equation}

\vspace{0.5cm}




Соотношение мощностей сравниваемых турбин легко получить, записывая выражение для мощностей на валу турбин (см. ф. (\ref{eq_fd2})). 

Мощность на валу первой турбины:

$$
   N_{\text{в}_1} = \rho g Q_1 \cdot H_1 \cdot \eta_{\text{турб}_1}
$$

Мощность на валу второй турбины:

$$
   N_{\text{в}_2} = \rho g Q_2 \cdot H_2 \cdot \eta_{\text{турб}_2}
$$

И соотношение мощностей:

$$
  \frac{N_{\text{в}_1}}{N_{\text{в}_1}} = \frac{\rho g Q_1 \cdot H_1 \cdot \eta_{\text{турб}_1}}  {\rho g Q_2 \cdot H_2 \cdot \eta_{\text{турб}_2}} = \frac{Q_1}{Q_2} \cdot \frac{H_1}{H_2} \cdot \frac{\eta_{\text{турб}_1}} {\eta_{\text{турб}_2}}
$$

\begin{equation}
\label{eq_fzx}
\boxed{
  \frac{N_{\text{в}_1}}{N_{\text{в}_1}} = \left( \frac{D_1}{D_2} \right)^2 \cdot \sqrt{\frac{\vphantom{\eta_{\text{турб}_2}} H_1}{ \vphantom{\eta_{\text{турб}_2}} H_2}} \cdot \sqrt{\frac{ \vphantom{1} \eta_{\text{турб}_1}}{\vphantom{1}\eta_{\text{турб}_2}}}     \cdot \frac{H_1}{H_2} \cdot \frac{\eta_{\text{турб}_1}} {\eta_{\text{турб}_2}}  
      }
\end{equation}

\vspace{0.5cm}

Формулы (\ref{eq_fzt}), (\ref{eq_fzv}) и (\ref{eq_fzx}) называются формулами подобия и очень широко используются. Поскольку гидравлический к.п.д. $\eta_{\text{турб}}$ изменяется мало, то часто при сравнении турбин его различиями пренебрегают, считая 

$$
   \frac{\eta_{\text{турб}_1}} {\eta_{\text{турб}_2}} = 1 \, .
$$


\vspace{1 cm}


 \begin{primer}
 \label{Primer3}
   \normalfontИсследования модели турбины диаметром \textnormal{ $D_{1_{\text{м}}} = 0.3$ м} при напоре \textnormal{$H_{\text{м}} = 3.5$ м} показали, что оптимальному режиму с наибольшим к.п.д. соответствует частота вращения \textnormal{$n_{\text{опт}_{\text{м}}} = 430 \, \frac{\text{об} }{\text{мин} }$}  и расход \textnormal{$Q_{\text{опт}_{\text{м}}} = 0.15 \frac{\text{м}^3 }{\text{с} }$}. Требуется найти частоту вращения \textnormal{$ n_{\text{опт}_{\text{н}}} $}, расход \textnormal{$ Q_{\text{опт}_{\text{н}}} $} и мощность \textnormal{$ N_{\text{опт}_{\text{н}}} $} натурной турбины того же типа, имеющей диаметр \textnormal{$ D_{1_{\text{н}}} = 5.0 $ м} и напор \textnormal{$ H_{\text{н}} = 90 $ м}.
 \end{primer}

 \textbf{Решение:} Принимаем, что турбина 1 -- это натурная турбина (индекс н), параметры которой надо найти. Турбина 2 -- модельная турбина (индекс м). По выражению (\ref{eq_fzt}), считая $\frac{\eta_{\text{турб}_1}} {\eta_{\text{турб}_2}} = 1$ находим $n_{\text{опт}_{\text{н}}}$:

$$
  \frac{n_{\text{опт}_{\text{н}}}}{n_{\text{опт}_{\text{м}}}} = \frac{D_{\text{м}}}{D_{\text{н}}} \cdot \sqrt{\frac{H_{\text{н}}}{H_{\text{м}}}} \, ,
$$

откуда

$$
   n_{\text{опт}_{\text{н}}} = \frac{D_{1_\text{м}} \cdot n_{\text{опт}_{\text{м}}} }{D_{1_\text{н}}} \cdot \sqrt{\frac{H_{\text{н}}}{H_{\text{м}}}}  = \frac{0.3 \cdot 430}{5} \cdot \sqrt{\frac{90}{3.5}} = 130.82 \; \frac{\text{об} }{\text{мин} }
$$

Аналогично, по выражению (\ref{eq_fzv}) определяем расход натурной турбины:

$$
    Q_{\text{опт}_{\text{н}}} =  Q_{\text{опт}_{\text{м}}} \cdot \left( \frac{D_{1_\text{н}} }{D_{1_\text{м}}} \right)^2  \cdot \sqrt{\frac{H_{\text{н}}}{H_{\text{м}}}} = 0.15 \cdot \left( \frac{5}{0.3} \right)^2 \cdot \sqrt{\frac{90}{3.5}} = 211.29 \; \frac{\text{м}^3 }{\text{с} }
$$

Поскольку величина расхода турбины и напора известна её мощность можно определить по выражению:

$$
   N_\text{в} = \rho g H Q_{\text{опт}_{\text{н}}} \cdot H_{\text{н}} \cdot \eta_{\text{турб}_{\text{н}}}
$$

Поскольку в условии задачи величина коэффициента полезного действия турбины не задана, примем его величину, исходя из того, что в оптимальном режиме $\eta_{\text{опт}}= $ 93 --- 95 \%. Принимаем $\eta_{\text{турб}_{\text{н}}} = 0.93$. Тогда величина мощности натурной турбины составит:

$$
   N_\text{в} = 1000 \cdot 9.806 \cdot 211.29 \cdot 90 \cdot 0.93 = 173\,419  \; \text{кВт}
$$


\vspace{1 cm}

Из рассмотренного примера можно увидеть, что при увеличении напора, при котором будет работать проектируемая турбина увеличится её мощность и частота вращения.


%%_________________________________________________________________________________________________________________________
%%_________________________________________________________________________________________________________________________
%%_________________________________________________________________________________________________________________________

\subsubsection{Приведённые параметры турбин.}

При сравнении турбин одного типа, работающих в одинаковых (подобных) режимах удобно использовать приведённые параметры турбин. Приведённые параметры характеризуют турбину некоторого типа, приведённую к общим, стандартным условиям. Принято давать показатели турбины, пересчитанные на напор $H = 1$ м и диаметр $D_1 = 1$ м. Эти параметры называются приведёнными (единичными) и обозначаются:

$n_I'$ -- приведённая частота вращения;

$Q_I'$ -- приведённый расход.

\vspace{0.5cm}

Получим выражения для определения приведённых параметров турбин. Для этого используем формулы подобия (\ref{eq_fzt}) и (\ref{eq_fzv}), считая что турбина 1 -- турбина, приведённая к стандартным условиям, её диаметр $D = 1$ м, напор, при котором она работает $H = 1$ м. А турбина 2 -- турбина, для которой имеются результаты модельных испытаний, и её приведённые параметры мы хотим определить (параметры работы турбины 2 запишем без индекса 2).

Тогда из выражения (\ref{eq_fzt}):

$$
   \frac{n_I'}{n} = \frac{D}{1} \cdot \sqrt{\frac{1}{H}} \cdot \sqrt{\frac{\eta_{\text{турб}}'}{\eta_{\text{турб}}}} \; ,
$$


откуда

\begin{equation}
\label{eq_fzy}
   n_I' = \frac{n \cdot D}{\sqrt{H}} \cdot \sqrt{\frac{\eta_{\text{турб}}'}{\eta_{\text{турб}}}}
\end{equation}

В приближённых вычислениях не учитывают изменения гидравлического коэффициента полезного действия, и тогда приближённое выражение для приведённой частоты вращения:

\begin{equation}
\label{eq_fzz}
   n_I' = \frac{n \cdot D}{\sqrt{H}}
\end{equation}

Аналогично, для приведённого расхода. Из выражения (\ref{eq_fzv}):

$$
   \frac{Q_I'}{Q} = \left( \frac{1}{D} \right)^2 \cdot \sqrt{\frac{\vphantom{\eta_{\text{турб}_2}} 1}{ \vphantom{\eta_{\text{турб}_2}} H}} \cdot \sqrt{\frac{ \vphantom{1} \eta_{\text{турб}}'}{\vphantom{1}\eta_{\text{турб}}}}
$$


\begin{equation}
\label{eq_fya}
   Q_I' = \frac{Q}{D^2 \sqrt{H}} \cdot \sqrt{\frac{\eta_{\text{турб}}'}{\eta_{\text{турб}}}} \; ,
\end{equation}

и приближённо:

\begin{equation}
\label{eq_fyb}
   Q_I' = \frac{Q}{D^2 \sqrt{H}}
\end{equation}


\vspace{0.5cm}

Значения приведённых параметров $n_I'$ и $Q_I'$ в подобных режимах сохраняются неизменными. Обычно их определяют по показаниям модельных испытаний.

Зная значения приведённых параметров, можно, используя формулы (\ref{eq_fzy}) и (\ref{eq_fya}), или их приближённые аналоги, определять фактические параметры натурных турбин при проектировании.



%%_________________________________________________________________________________________________________________________
%%_________________________________________________________________________________________________________________________
%%_________________________________________________________________________________________________________________________

\subsubsection{Коэффициент быстроходности.}

 \begin{opred}
    Коэффициент быстроходности $n_s$ численно равен частоте вращения турбины данного типа, но такого размера, что при напоре $H = 1$ м она развивает мощность 1 л.с. (лошадиная сила).
\end{opred}

Использование единицы мощности 1 л.с. $\approx$ 0.736 кВт -- исторически сложившийся приём.


%%**********************************************************************

Определим значение коэффициента быстроходности турбины, если известна её частота вращения n, напор H и мощность N. Для этого, как и в случае с приведёнными параметрами,  обратимся к формулам подобия, при этом полагая, что из сравниваемых турбин:


\vspace{0.5cm}

Турбина 1 -- это турбина, частота вращения которой равна коэффициенту быстроходности $n_s$, она работает при напоре $H=1$ м при этом развивает мощность $N_s = 1$ л.с. и её диаметр -- $D_s$\footnote{Следует обратить внимание, что диаметр этой турбины не принимается равным единице, это отличие в определении коэффициента быстроходности от приведённых параметров.}. 

\vspace{0.5cm}

Турбина 2 -- это турбина, у которой известны частота вращения n, мощность $N_{\text{[л.с.]}}$ и напор Н. Диаметр этой турбины D.

\vspace{0.5cm}

Для удобства вывода зависимостей примем в качестве размерности мощности лошадиные силы.

Тогда, в соответствии с формулой подобия (\ref{eq_fzt}) без учёта изменения к.п.д. (то есть, принимая $\frac{\eta_{\text{турб}_1}} {\eta_{\text{турб}_2}} = 1$):

\begin{equation}
\label{eq_fyc}
  \frac{n_s}{n} = \frac{D}{D_s} \cdot \sqrt{\frac{1}{H}} 
\end{equation}

откуда

$$
  \frac{D}{D_s} = \frac{n_s}{n \cdot \sqrt{\frac{1}{H}}} = \frac{n_s \cdot \sqrt{H}}{n}
$$

и

\begin{equation}
\label{eq_fyd}
  \frac{D_s}{D} = \frac{n}{n_s \cdot \sqrt{H}}
\end{equation}

\vspace{0.5cm}

В соответствии с формулой подобия (\ref{eq_fzx}) также без учёта изменения к.п.д.:


\begin{equation}
\label{eq_fye}
   \frac{1}{N_{\text{[л.с.]}}} = \left( \frac{D_s}{D} \right)^2 \cdot \sqrt{\frac{1}{H}} \cdot \frac{1}{H} 
\end{equation}

 

Подставим значение $\frac{D_s}{D}$ по выражению (\ref{eq_fyd}) в выражение (\ref{eq_fye}):

$$
  \frac{1}{N_{\text{[л.с.]}}} = \left( \frac{n}{n_s \cdot \sqrt{H}} \right)^2 \cdot \sqrt{\frac{1}{H}} \cdot \frac{1}{H} 
$$

$$
  \frac{1}{N_{\text{[л.с.]}}} = \frac{n^2}{{n_s}^2 \cdot H^2 \cdot \sqrt{H}}
$$

откуда 

$$
  n_s = \sqrt{\frac{n^2 \cdot N_{\text{[л.с.]}}}{H^2 \cdot \sqrt{H}}} = \frac{n}{H} \cdot \sqrt{\frac{N_{\text{[л.с.]}}}{\sqrt{H}}} \; ,
$$

или


\begin{equation}
\label{eq_fyf}
   n_s = \frac{n \cdot \sqrt{N_{\text{[л.с.]}}}}{H \sqrt[4]{H}}
\end{equation}


Если известна мощность в киловаттах $N_{\text{[кВт]}}$, или в ваттах $N_{\text{[Вт]}}$, то мощность в лошадиных силах определяется по выражениям:

$$
   N_{\text{[л.с.]}} = 1.36 \cdot N_{\text{[кВт]}} = 0.00136 \cdot N_{\text{[Вт]}} \; .
$$

Таким образом, выражение для коэффициента быстроходности:

\begin{equation}
\label{eq_fyj}
   n_s = \frac{n \cdot \sqrt{0.00136 \cdot N_{\text{[Вт]}}}}{H \sqrt[4]{H}}
\end{equation}



\vspace{1 cm}

Можно найти связь между коэффициентом быстроходности и приведёнными параметрами турбины $n_I'$ и $Q_I'$.

Мощность некоторой турбины в соответствии с (\ref{eq_fd2}):

$$
  N = \rho g Q \cdot H \cdot \eta_{\text{турб}}
$$

Выразим величину расхода, входящего в это выражение, через величину приведённого расхода $Q_I'$ по выражению (\ref{eq_fyb}) (без учёта изменения к.п.д.) и перепишем выражение для мощности турбины:

$$
   Q = Q_I' \cdot D^2 \cdot \sqrt{H}
$$

\begin{equation}
\label{eq_fyh}
   N = \rho g \cdot Q_I' \cdot D^2 \cdot \sqrt{H} \cdot H \cdot \eta_{\text{турб}}
\end{equation}

Частоту вращения рассматриваемой турбины также представим через приведённую частоту вращения по выражению (\ref{eq_fzz}):

\begin{equation}
\label{eq_fyi}
   n = \frac{n_I' \cdot \sqrt{H}}{D}
\end{equation}

Подставим полученные выражения (\ref{eq_fyh}) и (\ref{eq_fyi}) для частоты и мощности рассматриваемой турбины в выражение (\ref{eq_fyj}) для коэффициента быстроходности:

$$
   n_s = \frac{\frac{n_I' \cdot \sqrt{H}}{D} \cdot \sqrt{0.00136 \cdot \rho g \cdot Q_I' \cdot D^2 \cdot \sqrt{H} \cdot H \cdot \eta_{\text{турб}} }}{H \sqrt[4]{H}}
$$

$$
  n_s = n_I' \cdot \sqrt{0.00136 \cdot \rho \cdot g \cdot Q_I' \cdot \eta_{\text{турб}} }
$$

Откуда, считая плотность воды, равной $\rho = 1000 \; \frac{\text{кг}}{\text{м}^3}$ :

\begin{equation}
\label{eq_fyk}
   n_s \approx 3.65 \cdot n_I' \cdot \sqrt{Q_I' \cdot \eta_{\text{турб}}}
\end{equation}

Коэффициент быстроходности обычно определяется  для режима максимальной мощности. Он весьма полно характеризует свойства турбин, форму рабочего колеса, то есть виды и типы турбин.

Коэффициенты быстроходности различных видов турбин изменяются следующим образом:

\begin{table}[ht]
\caption{Величина коэффициента быстроходности для различных видов турбин.}
\label{tab_2s}
\centering

\begin{tabular}{|l|c|}
\hline
  \hspace{2 cm} Вид турбины & $n_s$\\
   & \\ 
\hline
  Осевые поворотно-лопастные & 1000 -- 450 \\
  Диагональные поворотно-лопастные & 500 -- 250 \\
  Радиально-осевые & 300 -- 80 \\
  Ковшовые & 50 -- 10 \\

\hline
\end{tabular}
\end{table}



Чем меньше коэффициент быстроходности, тем тихоходнее турбина, тем выше область её напоров.



%%_________________________________________________________________________________________________________________________
%%_________________________________________________________________________________________________________________________
%%_________________________________________________________________________________________________________________________

\subsubsection[Последовательность выбора турбины.]{Последовательность выбора турбины при проектировании ГЭС.}

Исходными данными для выбора турбины являются: мощность на валу одного гидроагрегата $N_{\text{в}}$ , величина расчётного напора $H_{\text{р}}$ и средневзвешенного напора $H_{\text{ср.взв.}}$

Мощность гидроагрегата может быть получена по величине установленной мощности проектируемой ГЭС -- $N_{\text{ГЭС}}$ , величина которой устанавливается на этапе определения гидропотенциала водотока. Также для определения мощности одного гидроагрегата необходимо задаваться количеством гидроагрегатов ГЭС. Количество гидроагрегатов $m$ зависит от мощности ГЭС и может приниматься различным, при этом необходимо учитывать возможность ремонта одного или нескольких гидроагрегатов без полной остановки выработки электроэнергии, то есть количество агрегатов должно быть, как минимум $m = 2$. При выборе числа агрегатов большее их количество приводит к установке сравнительно небольших агрегатов, меньших по стоимости, однако с увеличением количества агрегатов увеличивается площадь пространства, необходимого для их размещения (тем самым увеличивая затраты на производство бетонных работ), возрастает количество вспомогательного оборудования. Поэтому окончательный выбор числа агрегатов может быть сделан на основании сравнительного технико-экономического сравнения нескольких вариантов.

При известном числе агрегатов и величине установленной мощности ГЭС, мощность на валу одного гидроагрегата можно определить по выражению:

\begin{equation}
\label{eq_fyl}
   N_{\text{в}} =\frac{N_{\text{ГЭС}}}{ m \cdot \eta_{ \text{ген}}}
\end{equation}

где $N_{\text{ГЭС}}$ -- установленная мощность ГЭС;

$m$ --число гидроагрегатов;

$\eta_{ \text{ген}}$ -- к.п.д. генератора. В первом приближении можно принять $\eta_{ \text{ген}} = 0.
96 - 0.98 \,$, после выбора типа турбины эта величина может быть уточнена подбором конкретного генератора к выбранной турбине.

\vspace{0.5cm}

Величина расчётного напора $H_{\text{р}}$ устанавливается на этапе определения гидропотенциала, она учитывает колебания уровня воды при работе ГЭС,  сезонную сработку уровня. Также перед процедурой выбора типа турбины задаются ориентировочными значениями гидравлических потерь в подводящих частях здания ГЭС, на величину которых уменьшают статический (обусловленный разностью отметок уровней воды) напор (см. главу \ref{moshnost}). После выбора турбины и предварительного проектирования здания ГЭС эти параметры должны быть уточнены, и, при необходимости, процедура выбора турбины повторяется.

Величина средневзвешенного напора $H_{\text{ср.взв.}}$ учитывает как колебания уровней воды при работе ГЭС, так и изменения вырабатываемой ГЭС мощности в течение некоторого периода времени. Значение этой величины может быть задано в результате предварительного моделирования работы ГЭС.

$$
   H_{\text{ср.взв.}} = \frac{ \Sigma \left( N_{\text{гэс}} \cdot H \cdot \Delta t \right)}{\Sigma \left( N_{\text{гэс}} \cdot \Delta t \right)}
$$

где $N_{\text{гэс}}$ -- мощность ГЭС, которая изменяется во времени;

$H$ -- напор, срабатываемый ГЭС, также изменяется во времени;

$\Delta t$ -- некоторый рассматриваемый промежуток времени с определёнными значениями $N_{\text{гэс}}$ и $H$.

\vspace{0.5cm}

Итак,предположим что для выбора турбины известны значения $N_{\text{в}} \,$, $H_{\text{р}}$ и $H_{\text{ср.взв.}}$



\vspace{1cm}

Для каждого типа турбины по результатам модельных испытаний определяется область применения. Эта область применения ограничивает величины напора и мощности, при которых может применяться данный тип турбины. В пределах области применения турбина работает в режимах с наиболее высокими значениями к.п.д. При выходе из области применения турбина, конечно, будет вырабатывать электроэнергию, но с меньшей эффективностью.

Для удобства выбора гидравлических турбин применяют специальные сводные графики в координатах $H$ (напор), $N$ (мощность), на которых указаны области применения различных типов турбин.

Заданным значениям $N_{\text{в}} \,$ и $H_{\text{р}}$ на сводном графике может соответствовать несколько типов турбин, области применения которых перекрываются. Окончательный выбор типа турбины производится по итогам их сравнения. Сравнение типов турбин может производиться после определения их параметров: диаметра рабочего колеса $D_1$, частоты вращения $n$, высоты отсасывания $H_s$, величина которых находится для конкретных условий работы турбины (см. Пример \ref{Primer3}).

Далее ограничимся сравнением радиально-осевых и поворотно-лопастных типов турбин.

Области применения различных рассматриваемых типов турбин представлены на графике (см. Рис.\ref{img_ppl}), а также сведены в таблицу (\ref{tab_3}). При выборе турбин можно пользоваться как таблицей, так и сводным графиком (с той лишь разницей, что использование графика удобнее, но на нём невозможно отобразить всё разнообразие  типов турбин).




\begin{table}[ht]
\caption{Области применения различных типов турбин.}
\label{tab_3}

\centering
\begin{tabular}{|c|c|c|c|c|c|}

\hline
 
  &   & \multicolumn{2}{c|}{Напор $H$, м} & \multicolumn{2}{c|}{Диаметр $D_1$, м}\\

\cline{3-6}

\raisebox{6ex}[0cm][0cm]{Система турбин} & \raisebox{6ex}[0cm][0cm]{Тип турбины} &   \rotatebox{90}{ наибольший } & \rotatebox{90}{ наименьший } & \rotatebox{90}{ наименьший } & \rotatebox{90}{ наибольший }\\

\hline
 &        & \phantom{ooooo}  & \phantom{ooooo}  & \phantom{ooooo} & \phantom{ooooo}\\
Поворотно- & ПЛ7-ГК  & 7   & 3 &  & \\
лопастные & ПЛ10-ГК & 10  & 7 &  & \\
(горизонтальные & ПЛ15-ГК  & 15   & 10 & 3.55 & 10\\
капсульные) & ПЛ20-ГК & 20  & 15 &  & \\
 & ПЛ25-ГК & 25  & 20 &  & \\
\hline
\hline

           & ПЛ10-В  & 10   & 4 & 2 & 10.6\\
           & ПЛ15-В  & 15   & 10 & 2 & 10.6\\
           & ПЛ20-В  & 20   & 15 & 2 & 10\\
Поворотно- & ПЛ30-В  & 30   & 20 & 2 & 9.5\\
лопастные  & ПЛ40-В & 40  & 30 & 2 & 9.5\\
(вертикальные) & ПЛ50-В & 50   & 40 & 2 & 9\\
           & ПЛ60-В & 60  & 50 & 1.8 & 8.5\\
           & ПЛ70-В & 70  & 60 & 1.8 & 8.5\\
           & ПЛ80-В & 80   & 70 & 1.8 & 8.5\\
\hline
\hline
           & РО45-В  & 45   & 30 & 1.6 & 9\\
           & РО75-В  & 75   & 45 & 1.6 & 9\\
           & РО115-В  & 115   & 75 & 1.6 & 9\\
Радиально- & РО140-В  & 140   & 115 & 1.6 & 9\\
осевые     & РО170-В  & 170   & 140 & 1.6 & 9\\
(вертикальные) & РО230-В  & 230   & 170 & 1.6 & 9\\
           & РО310-В  & 310   & 230 & 1.25 & 6.7\\
           & РО400-В  & 400   & 310 & 1.25 & 5\\
           & РО500-В  & 500   & 400 & 1.25 & 4\\
           & РО600-В  & 600   & 500 & 1.25 & 3.55\\
\hline


\end{tabular}

\end{table}


\newpage

\begin{landscape}

\begin{figure} [ht]
  \center
  \includegraphics [scale = 0.83] {ppl}
  \caption{Сводный график областей применения крупных вертикальных поворотно-лопастных и радиально-осевых турбин.}
  \label{img_ppl}
\end{figure}

\end{landscape}


\newpage

Результатом выбора типа турбин является наименование нескольких типов турбин, каждый из которых рассматривается в качестве варианта для размещения в проектируемой ГЭС.

Для каждого такого варианта-типа турбины необходимо установить фактические параметры турбины, в соответствии с реальными условиями, поскольку информация о типе турбины предполагает лишь наличие приведённых параметров, определённых при модельных испытаниях. Проще говоря, теперь необходимо определить размеры турбины, подходящие для выработки заданной мощности. Это делается на основании закона подобия турбин.


Существуют отличия при определении фактических параметров турбин для разных типов турбин. Рассмотрим эту процедуру отдельно для двух типов.


\vspace{1.5cm}

%%_________________________________________________________________________________________________________________________
%%_________________________________________________________________________________________________________________________
%%_________________________________________________________________________________________________________________________

%%_________________________________________________________________________________________________________________________
%%_________________________________________________________________________________________________________________________
%%_________________________________________________________________________________________________________________________

\subsubsection{Выбор фактических параметров для радиально-осевых турбин.}



\textbf{п.1.} Устанавливаются значения необходимых приведённых параметров турбин:

\vspace{0.5cm}

$Q_{Ip}'$ -- максимальный приведённый расход $Q_{I\text{макс}}'$;

$n_{I\text{опт}}'$ -- оптимальная приведённая частота вращения;

$\overline{b_0}$ -- относительная высота направляющего аппарата;

$\sigma$ -- коэффициент кавитации.

\vspace{0.5cm}

Эти параметры характеризуют каждый из типов турбин. Значения этих параметров можно получить из документации, поставляемой производителями турбин, поскольку эти величины устанавливаются при проектировании турбин. 

Для радиально-осевых турбин, области применения которых представлены на рис.\ref{img_ppl} и в таблице \ref{tab_3}, необходимые параметры сведены в таблицу \ref{tab_4}.





\begin{landscape}


\renewcommand{\arraystretch}{1.4}
\begin{table}[ht]
\caption{Основные параметры радиально-осевых турбин.}
\label{tab_4}
\centering

\footnotesize
%\tiny
\begin{tabular}{|l|c|c|c|c|c|c|c|c|c|c|}



\hline
    & \multicolumn{10}{c|}{Тип гидротурбин} \\
\cline{2-11}


   \raisebox{1.5ex}[0cm][0cm]{Параметр} & \textbf{РО45} & \textbf{РО75} & \textbf{РО115} & \textbf{РО140} & \textbf{РО170} & \textbf{РО230} & \textbf{РО310} & \textbf{РО400} & \textbf{РО500} & \textbf{РО600} \\ 

\hline
  $Q_{I\text{макс}}'$ & 1350--1550 & 1150--1400 & 950--1150 & 850--1000 & 700--850 & 500--650 & 350--500 & 300--370 & 250--300 & 200--260 \\
\hline
  $Q_{I\text{опт}}'$  & 1150--1300 & 1000--1150 & 850--1000 & 750--900 & 550--700 & 400--550 & 300--400 & 200--300 & 180--250 & 150--200 \\
\hline
  $n_{I\text{опт}}'$  & 80--90 & 75--83 & 68--77 & 66--74 & 64--72 & 62--68 & 60--65 & 60--65 & 58--65 & 56--65 \\
\hline
  $\overline{b_0}$  & 0.35 & 0.3 & 0.25 & 0.25 & 0.2 & 0.16 & 0.12 & 0.1 & 0.08 & 0.06 \\
\hline
  $\sigma$ при $Q_{I\text{макс}}'$  & 0.15--0.24 & 0.13--0.21 & 0.1--0.17 & 0.09--0.15 & 0.07--0.13 & 0.05--0.09 & 0.04--0.07 & 0.04--0.065 & 0.04--0.065 & 0.04--0.065 \\

\hline
  $n_s$  & 320--390 & 280--340 & 230--290 & 210--260 & 180--230 & 150--190 & 120--160 & 110--140 & 100--120 & 90--120 \\
\hline
  $\overline{D_2}$  & 1.13 & 1.1 & 1.03 & 1.0 & 0.94 & 0.86 & 0.78 & 0.71 & 0.67 & 0.62 \\




\hline
\end{tabular}
\end{table}

где $Q_{I\text{макс}}'$ -- величина максимального приведённого расхода, $\frac{\text{л}}{\text{с}}$; 

$Q_{I\text{опт}}'$ -- величина оптимального приведённого расхода, $\frac{\text{л}}{\text{с}}$;

$n_{I\text{опт}}'$  -- оптимальная приведённая частота вращения, об/мин;

$\overline{b_0}$ -- относительная высота направляющего аппарата;

$\sigma$ -- значение коэффициента кавитации при максимальном расходе потока воды через турбину $Q_p' = Q_{I\text{макс}}'$;

$n_s$ -- коэффициент быстроходности турбины, об/мин;

$\overline{D_2}$ -- величина относительного диаметра (см. рис.\ref{img_pn} радиально-осевой турбины).


\end{landscape}



\textbf{п.2.} Определяется расход воды, проходящий через эту турбину.

Расход воды через турбину можно определить из выражения (\ref{eq_fd2}) мощности на валу турбины, задаваясь в первом приближении коэффициентом полезного действия турбины $\eta_{\text{турб}}$. Его величина может приниматься равной $\eta_{\text{турб}} = 0.85 - 0.96$.

После окончательного выбора типа турбины величина $\eta_{\text{турб}}$ может быть уточнена и, при необходимости, процедура расчёта, начиная с текущего пункта, повторяется.

$$
   Q = \frac{N_{\text{в}}}{\rho g \cdot H_p \cdot \eta_{\text{турб}}} \; ,
$$
где $Q$ -- расход воды, проходящий через турбину, $\frac{\text{м}^3}{\text{с}}$;

$N_{\text{в}}$ -- мощность на валу турбины, Вт;

$\eta_{\text{турб}}$ -- гидравлический к.п.д. турбины.


\vspace{0.5cm}

\textbf{п.3.} По значению приведённого расхода $Q_{Ip}'$ определяется требуемый диаметр рабочего колеса $D_1$.


В соответствии с выражением (\ref{eq_fyb}) диаметр рабочего колеса $D_1$ для данного рассматриваемого варианта можно определить по формуле:

$$
   D_1 = \sqrt{\frac{Q}{Q_{Ip} \cdot \sqrt{H_p}}} \; ,
$$

где $Q_{Ip}$ -- значение расчётного приведённого расхода для данного типа турбины (в качестве расчётного расхода (см. п.1) принимается максимальный приведённый расход, $\frac{\text{м}^3}{\text{с}}$;

$H_p$ -- величина расчётного напора, м;

$Q$ -- расход воды через турбину, $\frac{\text{м}^3}{\text{с}}$.

\vspace{1cm}

\textbf{п.4.} По величине вычисленного в предыдущем пункте диаметра рабочего колеса $D_1$  выбирается ближайшее унифицированное значение диаметра $D_1$.

Выбор одного из унифицированных диаметров обусловлен технологией изготовления рабочих колёс. Вместо того, чтобы каждую турбину изготавливать с индивидуальными размерами, проще использовать типовые проекты. Однако такие типовые решения могут отличаться на разных предприятиях, которые изготавливают турбины. Ряды унифицированных типоразмеров необходимо уточнять у производителей турбин, с которыми при проектировании ГЭС ведётся тесное взаимодействие. 

Поскольку, как уже говорилось ранее, диаметр $D_1$ рабочего колеса является определяющим (базисным) размером турбины, остальные размеры пропорциональны этому диаметру. Поэтому при выборе $D_1$ можно также установить величину остальных характерных размеров турбины\footnote{Вообще говоря, на этом шаге можно было бы практически закончить процедуру выбора размеров для этого варианта, поскольку в соответствии с утверждением о пропорциональности размеров, все остальные размеры можно определить через принятый диаметр $D_1$. Однако ГЭС работают при различных напорах, имеют различные компоновочные решения, которые в свою очередь налагают свои требования на размеры элементов ГЭС.}.

В качестве примера в таблице \ref{tab_5} приведены унифицированные значения диаметров рабочих колёс $D_1$ для различных типов турбин и соответствующие этим диаметрам величины, диаметр направляющего аппарата и число лопаток направляющего аппарата. 


\begin{table}[ht]
\caption{Основные параметры различных типов турбин.}
\label{tab_5}
\centering

\footnotesize

\begin{tabular}{|l|c|c|c|}

\hline
 
   & & & \\
 Система  & Диаметр рабочего  & Диаметр направляющего   & Число лопаток направ-\\
 гидротурбины & колеса $D_1$, м & аппарата $D_0$, м & ляющего аппарата $z_0$ \\
   & & & \\

\hline
    & 1.25 & 1.5 & 20\\
\cline{2-4}
    & 1.4 & 1.68 & 20\\
\cline{2-4}
    & 1.6 & 1.92 & 20\\
\cline{2-4}
    & 1.8 & 2.16 & 20\\
\cline{2-4}
\raisebox{5.5ex}[0cm][0cm]{Радиально-осевые} & 1.9 & 2.28 & 20  \\

\hline
\hline

    & 2.0 & 2.4 & 20\\
\cline{2-4}
    & 2.12 & 2.54 & 20\\
\cline{2-4}
    & 2.24 & 2.69 & 20\\
\cline{2-4}
    & 2.36 & 2.85 & 20\\
\cline{2-4}

    & 2.5 & 3.0 & 20\\
\cline{2-4}
    & 2.65 & 3.18 & 20\\
\cline{2-4}
    & 2.8 & 3.36 & 20\\
\cline{2-4}
    & 3.0 & 3.6 & 20\\
\cline{2-4}

    & 3.15 & 3.78 & 20\\
\cline{2-4}
    & 3.35 & 4.02 & 20\\
\cline{2-4}
    & 3.55 & 4.26 & 20\\
\cline{2-4}
    & 3.75 & 4.5 & 20\\
\cline{2-4}

    & 4.0 & 4.8 & 20\\
\cline{2-4}
    & 4.25 & 5.1 & 20\\
\cline{2-4}
    & 4.5 & 5.4 & 20\\
\cline{2-4}
    & 4.75 & 5.7 & 20\\
\cline{2-4}


    & 5.0 & 6.0 & 20\\
\cline{2-4}
    & 5.3 & 6.36 & 20\\
\cline{2-4}
    & 5.6 & 6.72 & 20\\
\cline{2-4}
    & 6.0 & 7.2 & 24\\
\cline{2-4}

    & 6.3 & 7.56 & 24\\
\cline{2-4}
    & 6.7 & 8.04 & 24\\
\cline{2-4}
    & 7.1 & 8.52 & 24\\
\cline{2-4}
    & 7.5 & 9.0 & 24\\
\cline{2-4}

\raisebox{35ex}[0cm][0cm]{Радиально-осевые} & 8.0 & 9.6 & 24  \\
\cline{2-4}
\raisebox{35ex}[0cm][0cm]{и} & 8.5 & 10.2 & 28  \\
\cline{2-4}
\raisebox{35ex}[0cm][0cm]{поворотно-лопастные} & 9.0 & 10.8 & 28  \\

\hline
\hline


    & 9.5 & 11.4 & 28\\
\cline{2-4}
    & 10.0 & 12.0 & 28\\
\cline{2-4}

\cline{2-4}
\raisebox{2.5ex}[0cm][0cm]{поворотно-лопастные} & 10.6 & 12.72 & 28  \\


\hline
\end{tabular}
\end{table}


\vspace{0.5cm}

Результатом выполнения  этого пункта процедуры выбора параметров турбины является установление следующих величин:

$D_1$ -- диаметр рабочего колеса (именно это унифицированное значение используется в последующих вычислениях), м;

$D_0$ -- диаметр направляющего аппарата (по его оси, то есть это диаметр окружности, проведённой через оси вращения лопаток направляющего аппарата, см. рис.\ref{img_pr}), м;

$z_0$ -- число лопаток направляющего аппарата.






\vspace{1cm}

\textbf{п.5.} Определяется фактическая частота вращения рабочего колеса.

Частоту вращения $n$ можно определить по выражению (\ref{eq_fzz}) по величине оптимальной приведённой частоты вращения $n_{I\text{опт}}'$ (значение $n_{I\text{опт}}'$ было установлено в п.1 выполняемой процедуры). 

\begin{equation}
\label{eq_fym}
  n = \frac{n_{I\text{опт}}' \cdot \sqrt{H_{\text{ср.взв.}}}}{D_1} \; ,
\end{equation}

где $H_{\text{ср.взв.}}$ -- величина средневзвешенного напора, м;

$n_{I\text{опт}}$ -- приведённая частота вращения турбины рассматриваемого типа, об/мин;

$D_1$ -- унифицированное значение диаметра рабочего колеса (его величина была определена в предыдущем пункте расчёта).

\vspace{0.5cm}

При определении частоты вращения в качестве напора рассматривается средневзвешенный напор $H_{\text{ср.взв.}}$ для учёта как колебаний уровня воды при работе ГЭС, так и изменения мощности за некоторый период работы. Величина $H_{\text{ср.взв.}}$ устанавливается на этапах проектирования, предшествующих выбору турбины,  предварительным моделированием работы ГЭС в течение некоторого периода. 

После определения частоты вращения необходимо принять в качестве проектной частоты вращения значение, соответствующее одному из рекомендованных для обеспечения установленной частоты переменного тока, вырабатываемого генератором. Ряд рекомендованных частот вращения приведён в конце параграфа, посвящённого гидрогенераторам (см. \ref{riad}).

Для выбора рекомендуемой частоты вращения величина $n$, вычисленная по формуле (\ref{eq_fym}), округляется в большую сторону до одного из значений рекомендованного ряда частот вращения.

Результат выполнения этого пункта является принятое для данного рассматриваемого варианта значение частоты вращения рабочего колеса -- одно из рекомендованных значений для выработки тока частотой 50 Гц.




\vspace{1cm}
\textbf{п.6.} Определяется высота отсасывания $H_s$.

Высота отсасывания может быть определена по выражению (\ref{eq_fze}) с введением некоторого запаса, обеспечивающего отсутствие кавитации в проточном тракте рабочего колеса.

$$
   H_s = 10.33 - \frac{\nabla}{900} - \frac{p_{\text{н.п.}}}{\rho g} - \sigma \cdot H_p - 1.5  \; ,
$$

где $\nabla$ -- отметка уровня воды в нижнем бьефе в балтийской системе высот, м;

$\frac{p_{\text{н.п.}}}{\rho g}$ -- высота водного столба, соответствующая давлению насыщенных паров, принимается по таблице \ref{tab_2}, м вод. ст.;

$\sigma$ -- коэффициент кавитации, характеризующий рассматриваемый тип турбины;

$H_p$ -- величина расчётного напора, м;

$1.5$ -- запас на высоту отсасывания, исключающий возникновение кавитации\footnote{Здесь показан способ исключения условий возникновения кавитации, отличный от применяемых в примерах \ref{Primer1} и \ref{Primer2}, в которых допустимая высота отсасывания уменьшалась путём увеличения коэффициента кавитации.}, м.

\vspace{0.5cm}

В случае отсчёта высоты отсасывания не от нижней кромки направляющего аппарата, а от горизонтальной оси симметрии, лопаток направляющего аппарата (находится в середине по высоте направляющего аппарата) высота отсасывания определяется по выражению:

$$
    {H_s}^{mid} = 10.33 - \frac{\nabla}{900} - \frac{p_{\text{н.п.}}}{\rho g} - \sigma \cdot H_p - 1.5 + \frac{b_0}{2}  \; ,
$$

где $b_0$ -- высота направляющего аппарата, м;

${H_s}^{mid}$ -- верхний индекс mid указывает на способ отсчёта высоты отсасывания, м.


\vspace{0.5cm}

Это конечный пункт выбора параметров для рассматриваемого варианта радиально-осевой турбины.

Результатами этой процедуры являются три основных параметра, подлежащих последующему сопоставлению с аналогичными параметрами других рассматриваемых вариантов турбин:

\begin{description}
\item[-] Тип турбины. 
\item[-] Диаметр рабочего колеса $D_1$ (унифицированное значение).
\item[-] Высота отсасывания $H_s$.
\item[-] Частота вращения рабочего колеса $n$ (принятая из рекомендованных).
\end{description}





%%_________________________________________________________________________________________________________________________
%%_________________________________________________________________________________________________________________________
%%_________________________________________________________________________________________________________________________

\subsubsection{Выбор фактических параметров для поворотно лопастных турбин.}


Выбор фактических параметров для поворотно-лопастных турбин отличается от аналогичной процедуры для радиально-осевых турбин тем, что фактические параметры определяются для двух различных режимов работы типа турбины, принятого к рассмотрению в каждом варианте.

Если принимать в качестве проектного какой-либо определённый режим работы, то в соответствии с теорией подобия режимов работы турбин, можно определить размеры турбины, которая будет вырабатывать определённую мощность при заданном напоре.

Для двух различных режимов размеры турбины, определённые таким образом, окажутся разными. То есть каждому из рассматриваемых режимов будет соответствовать свой вариант фактических параметров.

\vspace{0.5cm}

Итак, рассматривая один тип поворотно-лопастной турбины, получаем два варианта её размеров -- два способа её фактической реализации для выработки заданной мощности.

Каждому из двух рассматриваемых режимов соответствуют свои значения приведённых параметров.

\begin{center}
\textbf{Первый вариант поворотно-лопастной турбины рассматриваемого типа.}
\end{center}

Первый вариант соответствует оптимальному режиму работы турбины данного типа.

\vspace{0.5cm}

\textbf{п.1.1.} Устанавливаются значения необходимых приведённых параметров турбины(для рассматриваемого типа турбины эти значения выбираются из  таблиц \ref{tab_6} и \ref{tab_7}).

\vspace{0.5cm}

$Q_{I\text{опт}}'$ -- оптимальный приведённый расход;

$n_{I\text{опт}}'$ -- оптимальная приведённая частота вращения (принимается минимальной из диапазона значений в таблицах);

$\overline{b_0}$ -- относительная высота направляющего аппарата;

$\sigma_{\text{опт}}$ -- коэффициент кавитации, соответствующий оптимальному режиму работы турбины. Принимается минимальным из таблиц \ref{tab_6} и \ref{tab_7} (из графы Коэффициент кавитации $\sigma$ при $Q_{I\text{макс}}'$).

\vspace{0.5cm}

При проектировании эти данные получают у производителей турбин\footnote{Значения приведённых параметров снимаются с универсальных характеристик турбин, которые поставляются производителями турбин.}.


\begin{landscape}


\renewcommand{\arraystretch}{1.4}
\begin{table}[ht]
\caption{Основные параметры поворотно-лопастных турбин.}
\label{tab_6}
\centering

\footnotesize
%\tiny
\begin{tabular}{|l|c|c|c|c|c|c|c|c|c|}



\hline
    & \multicolumn{9}{c|}{Тип гидротурбин} \\
\cline{2-10}


   \raisebox{1.5ex}[0cm][0cm]{Параметр} & \textbf{ПЛ10} & \textbf{ПЛ15} & \textbf{ПЛ20} & \textbf{ПЛ30} & \textbf{ПЛ40} & \textbf{ПЛ50} & \textbf{ПЛ60} & \textbf{ПЛ70} & \textbf{ПЛ80}   \\ 

\hline
  $Q_{I\text{макс}}'$ & 2300--2500 & 2100--2350 & 1800--2250 & 1450--2000 & 1250--1700 & 1150--1500 & 1050--1300 & 950--1200 & 900--1100   \\
\hline
  $Q_{I\text{опт}}'$  & 1250--1450 & 1150--1350 & 1050--1250 & 1000--1200 & 1000--1200 & 950--1150 & 900--1050 & 850--1000 & 800--950    \\
\hline
  $n_{I\text{опт}}'$  & 165--185 & 150--165 & 138--150 & 125--135 & 115--126 & 110--120 & 105--116 & 102--110 & 100--105    \\
\hline
  $\overline{b_0}$  & 0.43 & 0.42 & 0.4 & 0.375 & 0.375 & 0.375 & 0.35 & 0.35 & 0.35    \\
\hline
  $\sigma$ при $Q_{I\text{макс}}'$  & 1.5--1.9 & 1.1--1.6 & 0.78--1.35 & 0.54--1.0 & 0.42--0.78 & 0.35--0.60 & 0.31--0.48 & 0.27--0.40 & 0.25--0.36    \\

\hline
  $n_s$  & 870--1010 & 750--880 & 640--780 & 520--600 & 450--570 & 410--510 & 370--460 & 340--420 & 320--380    \\
\hline
  $z_1$  & 3--4 & 3--4 & 4 & 4--6 & 5--6 & 7--8 & 7--8 & 8 & 8    \\

\hline
  $\overline{d_{\text{вт}}}$  & 0.35 & 0.35 & 0.37 & 0.4 & 0.44 & 0.47 & 0.51 & 0.56 & 0.60    \\



\hline
\end{tabular}
\end{table}

где $Q_{I\text{макс}}'$ -- величина максимального по кавитационным условиям приведённого расхода, $\frac{\text{л}}{\text{с}}$; 

\vspace{0.1cm}

$Q_{I\text{опт}}'$ -- величина оптимального приведённого расхода, $\frac{\text{л}}{\text{с}}$;

\vspace{0.1cm}

$n_{I\text{опт}}'$  -- оптимальная приведённая частота вращения, об/мин;

\vspace{0.2cm}

$\overline{b_0}$ -- относительная высота направляющего аппарата (минимальная величина);

$\sigma$ -- значение коэффициента кавитации при максимальном расходе потока воды через турбину $Q_p' = Q_{I\text{макс}}'$;

$n_s$ -- коэффициент быстроходности турбины, об/мин.

$z_1$ -- число лопастей рабочего колеса;

\vspace{0.2cm}

$\overline{d_{\text{вт}}}$ -- относительный диаметр втулки рабочего колеса.


\end{landscape}





\renewcommand{\arraystretch}{1.4}
\begin{table}[ht]
\caption{Основные параметры горизонтальных капсульных турбин (поворотно-лопастных).}
\label{tab_7}
\centering

\footnotesize
%\tiny
\begin{tabular}{|l|c|c|c|c|c|}



\hline
    & \multicolumn{5}{c|}{Тип гидротурбин} \\
\cline{2-6}


   \raisebox{1.5ex}[0cm][0cm]{Параметр} & \textbf{ПЛ7-ГК} & \textbf{ПЛ10-ГК} & \textbf{ПЛ15-ГК} & \textbf{ПЛ20-ГК} & \textbf{ПЛ25-ГК}   \\ 

\hline
  $Q_{I\text{макс}}'$ & 3200--3500 & 2800--3250 & 2400--3000 & 2000--2750 & 1700--2500     \\
\hline
  $Q_{I\text{опт}}'$  & 1800--2000 & 1700--1900 & 1600--1800 & 1500--1750 & 1450--1700  \\
\hline
  $n_{I\text{опт}}'$  & 150--190 & 145--180 & 140--170 & 135--165 & 135--160      \\
\hline
  $\overline{b_0}$  & 0.40 & 0.40 & 0.40 & 0.375 & 0.375    \\
\hline
  $\sigma$ при $Q_{I\text{макс}}'$  & 2.2--3.0 & 1.9--2.4 & 1.5--2.1 & 1.2--1.8 & 1.0--1.6    \\

\hline
  $n_s$  & 920--1230 & 850--1140 & 750--1020 & 660--920 & 610--850      \\
\hline
  $z_1$  & 3--4 & 3--4 & 4 & 4 & 4--5    \\

\hline
  $\overline{d_{\text{вт}}}$  & 0.35 & 0.35 & 0.36 & 0.37 & 0.40      \\

\hline
  $\alpha$  & 60--65 & 60--65 & 60--65 & 60 & 60   \\

\hline
  $\overline{D_{\text{к}}}$  & 1.1--1.15 & 1.15--1.2 & 1.15--1.2 & 1.2 & 1.2      \\

\hline
\end{tabular}
\end{table}

где $Q_{I\text{макс}}'$ -- величина максимального по кавитационным условиям приведённого расхода, $\frac{\text{л}}{\text{с}}$; 

\vspace{0.1cm}

$Q_{I\text{опт}}'$ -- величина оптимального приведённого расхода, $\frac{\text{л}}{\text{с}}$;

\vspace{0.1cm}

$n_{I\text{опт}}'$  -- оптимальная приведённая частота вращения, об/мин;

\vspace{0.2cm}

$\overline{b_0}$ -- относительная высота направляющего аппарата (минимальная величина);

$\sigma$ -- значение коэффициента кавитации при максимальном расходе потока воды через турбину $Q_p' = Q_{I\text{макс}}'$;

$n_s$ -- коэффициент быстроходности турбины, об/мин;

$z_1$ -- число лопастей рабочего колеса;

\vspace{0.2cm}

$\overline{d_{\text{вт}}}$ -- относительный диаметр втулки рабочего колеса.


$\alpha$ -- угол наклона оси поворота лопатки направляющего аппарата турбины к оси турбины, град;

\vspace{0.2cm}

$\overline{D_{\text{к}}}$ -- относительный диаметр капсулы.

\renewcommand{\arraystretch}{1.0}








\vspace{1.2cm}

\textbf{п.1.2.} Определяется расход воды, проходящий через турбину.

$$
  Q = \frac{N_{\text{в}}}{\rho \cdot g \cdot H_p \cdot \eta_{\text{турб}} } \; ,
$$

Это действие аналогично п.2 выбора фактический параметров для радиально-осевых турбин. Величина $\eta_{\text{турб}}$ также принимается равной  $\eta_{\text{турб}} = 0.85 - 0.96$ .


\vspace{0.5cm}
\textbf{п.1.3.} Определяется требуемый диаметр рабочего колеса.

$$
   D_1 = \sqrt{ \frac{Q}{Q_{I\text{опт}}' \cdot \sqrt{H_p}} } \; .
$$

\vspace{0.5cm}

\textbf{п.1.4.} Производится выбор ближайшего из унифицированных значений диаметров рабочего колеса, $D_1$, и соответствующих значений диаметра направляющего аппарата $D_0$, и число лопаток направляющего аппарата $z_0$.

\vspace{0.5cm}

Это действие аналогично п.4 выбора фактических параметров для радиально-осевых турбин. Унифицированные значения  также берутся из таблицы \ref{tab_5}.


\vspace{0.5cm}
\textbf{п.1.5.} Определяется фактическая частота вращения рабочего колеса $n$.

$$
   n = \frac{n_{I\text{опт}}' \cdot \sqrt{H_{\text{ср.взв.}}} }{D_1} \; .
$$

\vspace{0.5cm}

Принятое значение частоты вращения округляется в большую сторону в соответствии с рекомендованным рядом частот вращения, аналогично тому как это описано в п.5 выбора фактических параметров для радиально-осевых турбин.




\vspace{0.5cm}
\textbf{п.1.6.} Определяется высота отсасывания $H_s$.


Высота отсасывания определяется в соответствии с выражением \ref{eq_fze} с введением запаса, обеспечивающего отсутствие кавитации.

$$
   H_s = 10.33 - \frac{\nabla}{900} - \frac{p_{\text{н.п.}}}{\rho g} - \sigma_{\text{опт}} \cdot H_p - 1.5  \; ,
$$

Значения $\frac{p_{\text{н.п.}}}{\rho g}$ принимается по таблице \ref{tab_2}.

\vspace{0.5cm}

В случае, когда высота отсасывания горизонтальной поворотно-лопастной турбины отсчитывается не от верхней точки рабочего колеса, а от оси гидротурбины, величина высоты отсасывания определяется по выражению:

$$
   H_s^{Axe} = 10.33 - \frac{\nabla}{900} - \frac{p_{\text{н.п.}}}{\rho g} - \sigma_{\text{опт}} \cdot H_p - 1.5 - \frac{D_1}{2} \; ,
$$
где $D_1$ -- принятое унифицированное значение диаметра рабочего колеса, м;

$H_s^{Axe}$ -- верхний индекс Axe указывает на способ отсчёта высоты отсасывания.

\vspace{1cm}

Результатом выбора фактических параметров поворотно-лопастных турбин являются три основных параметра и тип рассмотренной в данном варианте турбины:

\begin{description}
\item[-] Тип турбины. 
\item[-] Диаметр рабочего колеса $D_1$ (унифицированное значение).
\item[-] Высота отсасывания $H_s$.
\item[-] Частота вращения рабочего колеса $n$ (принятая из ряда рекомендованных).
\end{description}









\newpage


\begin{center}
\textbf{Второй вариант поворотно-лопастной турбины рассматриваемого типа.}
\end{center}

Второй вариант соответствует режиму, при котором через турбину проходит поток с максимальным расходом. Отличие процедуры определения фактических параметров поворотно-лопастной турбины по второму варианту от аналогичной процедуры для первого варианта состоит лишь в различии значений приведённых параметров, которые характеризуют режим работы рассматриваемой турбины.

\vspace{0.5cm}

\textbf{п.2.1.} Устанавливаются значения приведённых параметров турбины.

$Q_{I\text{макс}}'$ -- максимальный приведённый расход;

$n_{I\text{макс}}'$ -- приведённая частота вращения, соответствующая максимальному приведённому расходу (Выбирается для рассматриваемого типа турбины по значениям из таблиц \ref{tab_6} и \ref{tab_7}: из графы $n_{I\text{опт}}'$ принимается максимальное значение);

$\overline{b_0}$ -- относительная высота направляющего аппарата;

$\sigma_{\text{опт}}$ -- коэффициент кавитации, соответствующий режиму работы турбины с максимальным расходом.

\vspace{0.5cm}

Указанные значения приведённых параметров\footnote{Следует ещё раз подчеркнуть, что эти значения приведённых параметров при проектировании ГЭС определяются по универсальным характеристикам турбины (снимаются с них), которые строятся производителями турбин при проведении модельных испытаний.} принимаются из таблиц \ref{tab_6} и \ref{tab_7}.



\vspace{0.5cm}

\textbf{п.2.2.} Определяется расход воды, проходящий через турбину.

$$
  Q = \frac{N_{\text{в}}}{\rho \cdot g \cdot H_p \cdot \eta_{\text{турб}} } \; ,
$$

где $\eta_{\text{турб}} = 0.85 - 0.96$ -- гидравлический к.п.д. турбины\footnote{Для корректного сравнения турбин значения коэффициента $\eta_{\text{турб}}$ в первом приближении следует принимать одинаковыми для всех рассматриваемых вариантов различных режимов и различных типов турбин.}. 

\vspace{0.5cm}

\textbf{п.2.3.} Определяется требуемый диаметр рабочего колеса.

$$
   D_1 = \sqrt{ \frac{Q}{Q_{I\text{макс}}' \cdot \sqrt{H_p}} } \; .
$$

\vspace{0.5cm}


\textbf{п.2.4.} Производится выбор ближайшего унифицированного значения диаметра рабочего колеса. Определяется диаметр направляющего аппарата $D_0$ и число лопаток направляющего аппарата $z_0$.

См таблицу \ref{tab_5}.

\vspace{0.5cm}

\textbf{п.2.5.} Определяется фактическая частота вращения рабочего колеса $n$.

$$
   n = \frac{n_{I\text{макс}}' \cdot \sqrt{H_{\text{ср.взв.}}} }{D_1} \; .
$$

Значение $n$ округляется в большую сторону в соответствии с рекомендованным (см. \ref{riad}) рядом частот вращений.


\vspace{2cm}
\textbf{п.1.6.} Определяется высота отсасывания $H_s$.

$$
   H_s = 10.33 - \frac{\nabla}{900} - \frac{p_{\text{н.п.}}}{\rho g} - \sigma_{\text{опт}} \cdot H_p - 1.5  \; ,
$$

$$
   H_s^{Axe} = 10.33 - \frac{\nabla}{900} - \frac{p_{\text{н.п.}}}{\rho g} - \sigma_{\text{опт}} \cdot H_p - 1.5 - \frac{D_1}{2} \; ,
$$

\vspace{0.5cm}

Результатом выполнения процедуры определения фактических параметров поворотно-лопастной турбины по второму варианту также являются параметры:


\begin{description}
\item[-] Тип турбины. 
\item[-] Диаметр рабочего колеса $D_1$ .
\item[-] Высота отсасывания $H_s$.
\item[-] Частота вращения рабочего колеса $n$.
\end{description}





\vspace{1.5cm}

После выбора фактических параметров рассмотренных вариантов типов турбин, а также различных режимов их работы, производится выбор одного из этих вариантов для размещения в здании ГЭС.

На первом этапе сравнения предпочтение отдаётся вариантам турбины с наибольшей высотой отсасывания (чем больше высота отсасывания, тем меньший объём бетонных работ потребуется в связи с уменьшением заглубления здания ГЭС). При этом из рассмотрения сразу исключаются варианты с высотой отсасывания $H_s < -3.0$ м.

На втором этапе сравнения отдают предпочтение турбинам с наименьшим диаметром рабочих колёс $D_1$, однако также принимается во внимание частота вращения рабочего колеса $n$ (чем выше частота вращения -- тем быстрее будет происходить износ опорных деталей, вызывая более частую потребность в проведении текущего ремонта).

Указанные критерии сравнения, конечно, не являются строгими. Окончательный выбор типа и размеров турбины может быть выполнен на основании технико-экономического сравнения.

\vspace{0.5cm}

Упрощённо, сравнение рассмотренных вариантов:

\begin{description}
\item[1. по $H_s$]: Чем больше величина $H_s$ -- тем лучше. 
\item[2. по $D_1$]: Чем меньше величина $D_1$ -- тем лучше.
\item[3. по $n$]: Чем меньше величина $n$ -- тем лучше.
\end{description}

Причём, высота отсасывания наиболее значимый параметр.






